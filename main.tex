%
% Master thesis template for Ghent University (2018)
%
%
%  !!!!!!!!!!!!!!!!!!!!!!!!!!!!!!!!!!!!!!!!!!!!!!!!!!!!!!!!!!!!
%  !!  MAKE SURE TO SET lualatex OR xelatex AS LATEX ENGINE  !!
%  !!!!!!!!!!!!!!!!!!!!!!!!!!!!!!!!!!!!!!!!!!!!!!!!!!!!!!!!!!!!
%  !! For overleaf:                                          !!
%  !!     1. click gear icon in top right                    !!
%  !!     2. select `lualatex` in "latex engine"             !!
%  !!     3. click "save project settings"                   !!
%  !!                                                        !!
%  !!!!!!!!!!!!!!!!!!!!!!!!!!!!!!!!!!!!!!!!!!!!!!!!!!!!!!!!!!!!
%
%
%  History
%    2014         Doctoral Thesis of Bruno Volckaert
%    2017         Adapted to master thesis by Jerico Moeyersons
%    2018         Cleanup by Merlijn Sebrechts
%
%  Latest version
%    https://github.com/galgalesh/masterproef-template
%
\documentclass[11pt,a4paper,twoside, openany]{book}
\usepackage[a4paper,includeheadfoot,margin=2.50cm]{geometry}

\setlength{\parindent}{0cm}           % indent of the first sentence of a paragraph
\setlength{\parskip}{1em}             % space between paragraphs
\renewcommand{\baselinestretch}{1.2}  % stretch horizontal space between everything

\usepackage{graphicx}
\graphicspath{{images/}}
\usepackage{pdfpages}
\usepackage{enumerate}
\usepackage{float}
\usepackage{caption}
\usepackage{subcaption}
\usepackage[toc,page]{appendix}

\usepackage{minted}                                    % for modern code highlighting
\newenvironment{code}{\captionsetup{type=listing}}{}   % To get multiline code fragments working: https://tex.stackexchange.com/a/53540/72273

\PassOptionsToPackage{hyphens}{url}
\usepackage{hyperref}
\usepackage{url}

\usepackage{quotchap}              % For the fancy quotes next to the chapter titles

\usepackage[numbers]{natbib}       % For bibliography; use numeric citations
\bibliographystyle{IEEEtran}
\usepackage[nottoc]{tocbibind}     % Put Bibliography in ToC

%
% Defines \checkmark to draw a checkmark
%
\usepackage{tikz}
\def\checkmark{\tikz\fill[scale=0.4](0,.35) -- (.25,0) -- (1,.7) -- (.25,.15) -- cycle;}

%
% For tables
%
\usepackage{booktabs}
\usepackage{array}
\usepackage{ragged2e}  % for '\RaggedRight' macro (allows hyphenation)
\newcolumntype{L}[1]{>{\raggedright\let\newline\\\arraybackslash\hspace{0pt}}m{#1}}
\newcolumntype{C}[1]{>{\centering\let\newline\\\arraybackslash\hspace{0pt}}m{#1}}
\newcolumntype{R}[1]{>{\raggedleft\let\newline\\\arraybackslash\hspace{0pt}}m{#1}}

%
% Support for splitting Dutch words correctly
%
\usepackage{polyglossia}
\setdefaultlanguage[babelshorthands=true]{dutch}

% Manually specify additional hypnations for words
\hyphenation{ten-ants appli-caties Open-Stack-Emu cloud-besturings-sys-temen besturings-sys-temen Dev-Stack Volckaert}

%
% Translated strings. If these aren't set, the English words are used.
%
\addto\captionsenglish{%
  \renewcommand{\contentsname}%
    {Inhoudsopgave}%
}
\renewcommand\appendixtocname{Bijlagen}
\renewcommand\appendixpagename{Bijlagen}
\renewcommand{\listoflistingscaption}{Lijst van listings}

% Added by Stijn Brysbaert
% usepackage for markdown
\usepackage[hashEnumerators,smartEllipses]{markdown}
% usepackage for todos
\usepackage[colorinlistoftodos]{todonotes}
\usepackage{verbatim}
% usepackage for glossaries
\usepackage[utf8]{inputenc}
\usepackage[acronym, toc]{glossaries}
% strikethrough
\usepackage{ulem}

\makeglossaries
\newacronym{www}{www}{World Wide Web}

\newacronym{w3c}{W3C}{World Wide Web Consortium}

\newacronym{rdf}{RDF}{Resource Description Framework}

\newacronym{lod}{LOD}{Linked Open Data}

\newacronym{maas}{MaaS}{Mobility as a Service}

\newacronym{uri}{URI}{uniform resource identifier}

\newacronym{id}{ID}{identificator}

\newacronym{eif}{EIF}{European Interoperability Framework}
\makeglossaries
\newglossaryentry{maasaanbieder}
{
    name=MaaS-aanbieder,
    description={Een organisatie die een \acrshort{maas} platform aanbiedt}
}

\newglossaryentry{mobop}{
    name=mobiliteitsoperator,
    plural=mobiliteitsoperatoren,
    description={Een organisatie die van mobiliteit een service maakt en deze aanbiedt aan reizigers}
}

\newglossaryentry{deelmob}{
    name = deelmobiliteit,
    description = {Het delen van een fiets, auto, brommer, ... met andere mensen zodat kosten zoals aankoop en onderhoud kunnen verdeeld worden onder de gebruikers. Kan aangeboden worden als abonnement of pay as you go waarbij voor beide opties de kosten meestal oplopen naar gelang de tijd dat je het voertuig gebruikt}
}

\newglossaryentry{fietsdeelop}{
    name = fietsdeeloperator,
    plural = fietsdeeloperatoren,
    description = {Een combinatie van \gls{mobop} en \gls{deelmob} waarbij het vervoermiddel een fiets is}
}

\newglossaryentry{ontologie}{
    name = vocabularium,
    plural = vocabularia,
    description = {Een vocabularium definieert concepten en relaties (ook termen genoemd) om zaken binnen een bepaalt gebied te omschrijven en weer te geven~\footnote{\url{https://www.w3.org/standards/semanticweb/ontology}}}
}

\newglossaryentry{RDF vocabularium}{
    name = RDF vocabularium,
    description = {Zie \gls{ontologie.}}
}

\newglossaryentry{deeleconomie}{
    name = deeleconomie,
    description = {De deeleconomie is een socio-economisch systeem waarin delen en collectief consumeren centraal staat. Het gaat om gezamenlijk creatie, productie, distributie, handel en consumptie van goederen en diensten\footnote{\url{https://nl.wikipedia.org/wiki/Deeleconomie}}}
}

\newglossaryentry{dock}{
    name = dock,
    plural = docks,
    description = {Een gereserveerde plaats waar exact één voertuig in kan worden geparkeerd. In sommige gevallen is een dock enkel en alleen bruikbaar voor het parkeren van een specifiek vervoermiddel. Voertuigen van een ander type dan waarvoor het dock is bedoeld, kunnnen hier niet worden geparkeerd}
}

% Added by supervisor Pieter Colpaert to make comments
\newcommand{\pc}[1]{\noindent\textcolor{red}{\{Pieter says: #1{\bf \}}}}
%
% Set the title and your name
%
\title{Het publiceren van gegevens rond fietsdeeloperatoren als Linked Data}
\author{Stijn Brysbaert}

%
%  END OF HEADER
%  The actual latex document content starts here.
%
\begin{document}

\includepdf{voorblad.pdf}             % Front matter
\newpage\thispagestyle{empty}\mbox{}  % White page
\thispagestyle{empty}    % Don't show page number

\begin{center}
\textbf{Dankwoord}
\end{center}

Deze masterproef is het eindwerk van mijn masteropleiding en daarmee ook het einde van mijn hoger onderwijs traject. Na het behalen van een professionele bachelor startte ik een schakeljaar industriële wetenschappen afstudeerrichting informatica om daar aansluitend de bijhorende masteropleiding af te leggen. Door dit te combineren met de rol als groepsleider van een scoutsgroep met 360 leden werd dit een intensieve periode van drie en een half jaar. Gedurende die periode stond ik er gelukkige nooit alleen voor. Ik kon rekenen op veel mensen rondom mij die me ondersteunden op vlak van zowel scouting als studies. De ruimte dat dit dankwoord mag innemen is helaas te klein om al die namen te vernoemen. Toch had ik graag mijn ouders vermeld en hun bedankt mij de mogelijkheid te geven te kunnen studeren wat, wanneer en hoe ik zelf wilde. Ik ben me ervan bewust dat dit een uitzonderlijke kans is. Dankjewel iedereen.

Deze masterproef kwam tot stand met de onmisbare hulp van een aantal personen. Allereerst had ik mijn promotoren prof. dr. ir. Ruben Verborgh en dr. Pieter Colpaert willen bedanken mij de kans te geven me te kunnen verdiepen in de interessante en veel belovende wereld van Linked Data. Daarbovenop op wil ik Pieter extra bedanken voor zijn enthousiasme en gedrevenheid in het onderwerp. Mijn doel iets interessant te maken van mijn masterproef werd ook zijn doel. Altijd beschikbaar en paraat met kritische ideeën en feedback. Ook mijn begeleiders Brecht Van de Vyvere, Dylan Van Assche, Julian Andres Rojas Melendez en Harm Delva wil ik bedanken voor hun vooral technische ondersteuning via ons mattermost kanaal op ieder moment van de dag. Speciale dank gaat uit naar mijn broer Maarten\footnote{\url{https://www.linkedin.com/in/maarten-brysbaert}} die me op zeer korte tijd een demo hielp in elkaar steken waarmee ik mijn contributies kon visualiseren tijdens presentaties.

Om af te sluiten had ik graag mijn vriendin Feuille Boi bedankt voor haar onvermoeibare steun, ook tijdens moment waarop dat niet zo vanzelfsprekend was. Ook mijn zussen die altijd blij waren me te zien als ik thuis kwam waren een steun. Daarnaast wil ik mijn huisgenoten Aileen Bonne en Wiebe Vermiesch bedanken voor de leuke momenten tussendoor en andere scoutsvrienden voor de ontspannende wandelingen mét social distancing uiteraard. Extra dank gaat uit naar Aileen voor de zelfreflectie momenten ten behoeve van mijn masterproef.

Bedankt allemaal!
\newline
Stijn Brysbaert          % Word of thanks
\newpage\thispagestyle{empty}\mbox{}  % White page
% \listoftodos
Abstract

% Context
There are different ways to publish mobility related data causing these datasets to be non-interoperable.
% Need
A unified data model that fosters creation of reusable identifiers for mobility infrastructure components, is required to increase data interoperability.
% Task
Following the principles of Linked Data, a proposal for an extensible RDF vocabulary that captures the straightforward object properties of bicycle sharing data is designed.
% Object
The vocabulary is focusing on the bicycle sharing use case so data of Flemish bicycle sharing operators can be republished as Linked Data.
% Findings
The OSLO mobiliteit: trips \& aanbod ontology of the Flemish government is a candidate to expand with properties defined in the GBFS specification to satisfy the needs of operators like Blue Bike and Velo.
% Conclusion
Mobility operators can republish their data in our developed vocabulary, but some of them can't invest a lot of time and money in republishing their data in a provided data model. Therefore, an intermediate party is needed to republish their data as Linked Data.
% Perspective
During following phases of research, the RDF data model is ready to be expanded to future mobility operators and become a single, all-encompassing and interoperable data model for mobility related data.
%\includepdf[pages={-}]{abstract.pdf}  % Extended Abstract

\tableofcontents                      % Table of Contents
\listoffigures                        % List of figures
\listoftables                         % List of tables
\listoflistings                       % List of listings (code fragments)
\printglossary
\printglossary[type=\acronymtype]

%
% Include the main chapters of the thesis below
%


\chapter{Introductie}
\label{chap:interoperabele_gegevens}
Gegevens die worden gemodelleerd vanuit één enkel perspectief kunnen niet worden gecombineerd of geïntegreerd met andere informatiebronnen of toepassingen en business processen~\cite{interoperability}. Gegevens uit het ene ICT systeem zullen dus niet zomaar bruikbaar zijn in een ander systeem. Bijvoorbeeld, busmaatschappij De Lijn haalt informatie op over wegenwerken uit de database van het Vlaamse Agentschap Wegen en Verkeer. Die informatie kan nuttig zijn om eventuele wegenwerken die het busverkeer hinderen op te sporen. De Lijn kan op basis van deze informatie aanpassingen aanbrengen in de dienstregeling zodat die buslijn\footnote{Busmaatschappij De Lijn gebruikt de term 'lijn' om een bepaalt traject van A naar B aan te duiden.} bedient blijft. In een niet-interoperabel scenario, zal de busmaatschappij voor ieder resultaat uit de wegenwerken database manueel moeten nagaan of lopende werken één of meerdere lijnen doorkruisen of niet. In een wereld waar gegevens tussen de publieke en private sector wel interoperabel zijn, kunnen de systemen van de ene partij overweg met de gegevens van een andere partij. Het computersysteem van De Lijn zal de resultaten van een query, uitgevoerd op de database van Agentschap Wegen en Verkeer, automatisch kunnen interpreteren en het personeel van De Lijn kunnen waarschuwen voor potentiële hinder op een bepaalt deel van een lijn. Daarna kunnen verdere, eventueel manuele, acties worden ondernomen.

Dankzij interoperabiliteit tussen gegevens, krijgen organisaties de mogelijkheid om informatie en kennis te delen op hun eigen manier door middel van gegevensuitwisseling tussen ICT systemen. Het \acrfull{eif} beschrijft een model voor interoperabiliteit dat toepasbaar is op Europese digitale publieke diensten. Met dit programma wilt Europa bouwen aan een naadloze gegevensdoorstroming binnen Europese publieke diensten zoals het Vlaams Agentschap voor Wegen en Verkeer in het voorbeeld hierboven~\cite{neweif}. In de volgende secties van dit hoofdstuk wordt er dieper ingegaan op wat die interoperabiliteit precies is en hoe die kan worden geïmplementeerd.

\section{Semantische interoperabiliteit}
\label{sec:semantische_interoperabiliteit}
Semantische interoperabiliteit gaat over de betekenis en relaties tussen gegevens. Het behandelt zowel syntactische als semantische aspecten. Interoperabiliteit verzekert dat het formaat (syntactisch) en de inhoud (semantisch) van informatie dat werd verzonden, bewaart blijft en wordt begrepen door de ontvanger zoals bedoeld door de verzender. Met andere woorden: Wat werd verzonden, is wat werd begrepen~\cite{neweif}.

Het is niet voor de hand liggend dat iets wordt geïnterpreteerd zoals het werd bedoeld. Als de verzender spreekt over het Sint-Pietersplein, kan dit worden geïnterpreteerd als meerdere verschillende plaatsen of objecten. De ene ontvanger denkt hierbij aan het Sint-Pietersplein in Gent, maar iemand anders denkt direct aan het plein in Vaticaanstad. Parking P10 in Gent heeft dezelfde naam, die bevindt zich onder het Sint-Pietersplein in Gent. Dit probleem doet zich ook voor bij gegevens in een database.

Met behulp van metadata kan duidelijk worden gemaakt wat er precies wordt bedoeld met een bepaalt object. Door relaties tussen objecten te omschrijven kan er nog meer context worden gecreëerd waardoor zowel mens als computer beter kan begrijpen wat er wordt bedoeld door de zender. Om relaties tussen objecten mogelijk te maken zijn er herbruikbare identificatoren nodig die een object uniek identificeren.

\section{Herbruikbare identificatoren}
\label{sec:herbruikbare_ids}
De databases van publieke sectoren bevatten miljoenen objecten. Het is mogelijk dat een zelfde object gebruikt wordt door meerdere verschillende partijen. Neem bijvoorbeeld een fietsenstalling aan een busstation waar reizigers hun fiets kunnen plaatsen als ze de bus nemen, maar waar ook deelfietsen van bijvoorbeeld Velo kunnen worden ontleend. In dat geval wordt het gebruik en de verantwoordelijkheid van dat object uitgebreid naar meerdere partijen: meerdere \glspl{mobop} en diensten van de stad/gemeente die bijvoorbeeld in onderhoud en reparaties voorziet. Gezien dat gedeeld gebruik hebben alle partijen toegang nodig tot de gegevenssets waarin de parameters van dat object lees- en schrijfbaar zijn. Zo een situatie vraagt om interoperabiliteit van gegevens en herbruikbare identificatoren die deze objecten aanduiden. 

Neem als voorbeeld de situatie in figuur~\ref{fig:busstation_labels}. Een busmaatschappij gebruikt de \acrfull{id} van een busstation om aan te duiden waar een bepaalde bus zal stoppen. Datzelfde ID kan door de operator achter de Velo-fietsen worden hergebruikt om aan te duiden dat er een fietsenstalling bij dat busstation te vinden is. Als er elektriciteitsvoorzieningen nodig zijn tot bij het object, kunnen elektriciens door middel van het busstation ID en andere technische informatie precies weten hoe ze te werk moeten gaan. Zo kunnen er bijvoorbeeld verlichtings- en oplaadpunten voor elektrische fietsen worden voorzien en informatie hierover gekoppeld worden aan de ID's van het busstation en fietsenstalling.

\begin{figure}[h]
	\centering
	\begin{subfigure}{\textwidth}
		\centering
		\centerline{
			\includegraphics[scale=0.35]{images/busstation_labels.png}
		}
	\end{subfigure}
	\caption{Parameters van een infrastructuur object, gebaseerd op een figuur van istockphoto.com}
	\label{fig:busstation_labels}
\end{figure}

\section{Open Standaarden voor Linkende Organisaties}
\label{sec:OSLO}
Het Vlaams interoperabiliteitsprogramma Open Standaarden voor Linkende Organisaties (OSLO)\footnote{\url{https://data.vlaanderen.be}} is een initiatief van de Vlaamse overheid om hergebruik van gegevens te stimuleren. Het is een van de stappen die Vlaanderen neemt richting een Open Overheid met als doel een nauwere samenwerking te creëren tussen overheid en de privé sector.
Een vorige stap naar open data was het openbaar beschikbaar maken van het Grootschalig Referentiebestand (GRB). Het GRB is een digitale topografische referentiekaart van Vlaanderen waarop alle gebruikers eigen geografische gegevens kunnen aanduiden. Deze locatiegegevens van gebouwen, percelen, wegen en hun inrichting, waterlopen, spoorbanen en het wegennetwerk identificeren miljoenen objecten in Vlaanderen. Dit referentiebestand wordt gebruikt voor onder andere dispatchingtool bij hulpdiensten en het lokaliseren en intekenen van ondergrondse kabels en leidingen~\cite{wat_is_grb}. Sinds het GRB openbaar beschikbaar werd via een gegevensportaal\footnote{\url{http://opendata.vlaanderen.be}} is het gebruik ervan toegenomen van 600 aanvragen naar 2500 per maand~\cite{grb_OD}.

Desondanks de nuttige toepassingen dat het systeem brengt, ondervinden heel wat gebruikers (private partners, bevolking en publieke administratiediensten) moeilijkheden bij het verbinden met en interpreteren van deze open data bronnen. Deze problemen met de interoperabiliteit van gegevens veroorzaken strubbelingen bij het hergebruiken van deze publieke sectorinformatie~\cite{opengovernment}. Om de vraag naar interoperabele gegevens te beantwoorden werd het OSLO programma opgestart dat verder bouwt op de principes van Linked Data en voldoet aan de aanbevelingen uit het \acrshort{eif}.

Het OSLO programma legt de betekenis vast van concepten, woorden en definities en hoe ze te structureren zijn in databases of softwarepakketten. Dankzij deze afspraken, omschreven in datastandaarden, kunnen semantische discussies en misverstanden worden vermeden~\cite{OSLO_handleiding}.

\subsection{OSLO services}
\label{sec:oslo_services}
Informatie Vlaanderen biedt verschillende services aan waarmee het organisaties, in zowel de publieke als privé sector, in het OSLO traject helpt. Samen met de organisatie in kwestie onderzoekt het OSLO-team in hoeverre de aangeleverde informatiemodellen kunnen afgestemd worden op bestaande OSLO-vocabularia. Nadat de noden voor een gegevensmodel in kaart gebracht werden, zal het team een datastandaard (of OSLO-vocabularium) ontwikkelen volgens `Proces \& Methode' van OSLO\footnote{\url{https://data.vlaanderen.be/cms/Proces_en_methode_voor_de_erkenning_van_datastandaarden_v1.0.pdf}}, dat zijn richtlijnen voor het ontwikkelen van nieuwe standaarden binnen OSLO. Nadat er een gepaste datastandaard ontwikkeld of een reeds bestaande standaard toegewezen werd, wordt er ook ondersteuning geboden bij het implementatieproces.

\subsection{Verhoogde interoperabiliteit met OSLO}
Het doel van OSLO is om de gegevensoverdracht tussen verschillende organisaties te versnellen en automatiseren. Dit is nodig omdat de overheid meer dan duizend diensten aanbiedt aan burgers en bedrijven~\cite{OSLO_handleiding}. Deze samenwerking kost tijd en geld doordat gegevens niet interoperabel zijn. Dankzij de hulp van de experten binnen Informatie Vlaanderen, die OSLO mogelijk maken, wordt er aan de nood van interoperabele gegevens tegemoetgekomen. Gegevens worden met OSLO eenduidig gedefinieerd en kunnen beter worden hergebruikt met als resultaat meer samenhang tussen informatie en bijgevolg betere begrijpbaarheid en vindbaarheid ervan.

De huidige doelstelling bestaat erin om zoveel mogelijk gegevenssets te (her)publiceren als \acrshort{lod} met behulp van het OSLO programma. Gegevens rond nieuwe projecten kunnen vanaf nul worden opgebouwd met gestandaardiseerde gegevensmodellen. Reeds bestaande gegevenssets moeten worden omgevormd, zodat het gebruikmaakt van die standaard modellen. Hiervoor zijn tools nodig die verder in deze scriptie worden besproken.

\section{Open Data}
\label{chap:intro}

Tim Berners-Lee, de grondlegger van het \acrfull{www} en oprichter van het \acrfull{w3c}, pleit al jaren voor ``raw data''\footnote{video:  \url{https://www.ted.com/talks/tim_berners_lee_the_next_web} op 10'40''}. Daarmee vraagt hij aan instellingen zoals overheden, onderzoekscentra en bedrijven om hun gegevens openbaar op het internet beschikbaar te stellen zodat het toegankelijk is voor iedereen om er onderzoek op uit te voeren~\cite{tedtalk}. Het openbaar maken van data en het zo ter te beschikking stellen voor iedereen zou wereldverbeterende inzichten moeten opleveren zoals economische groei, efficiënter gebruik van resources en een beter leefwereld voor burgers~\cite{tedtalka}. 
Die inzichten kunnen op een efficiënte manier worden verworven door die open data interpreteerbaar te maken voor machines. Om het Web ook machine-interpreteerbaar te maken werd het concept `Semantisch Web' geïntroduceerd door Tim Berners-Lee. Volgens hem is de meeste inhoud van het web bedoeld voor mensen om te lezen. Computers kunnen goed webpagina's parsen voor layout en routine werk uitvoeren, maar ze hebben geen betrouwbare manier om semantische inhoud te verwerken~\cite{tim_semanticweb}. Computers kunnen, met andere woorden, de bedoeling van de inhoud van een pagina niet interpreteren.

\section{Semantisch Web}
\label{semantisch_web}
Het idee van het Semantisch Web is "om een een Web te weven dat niet enkel documenten aan elkaar linkt, maar ook de betekenis van informatie in die documenten herkent" \cite{frauenfelder}.
Dit Semantisch Web moet gezien worden als een uitbreiding van het huidige Web door een gemeenschappelijke structuur aan te brengen aan de inhoud van webpagina's om zo bij te dragen aan een betere samenwerking tussen mens en machine.~\cite{kuck} Om tot dit Semantisch Web te komen omschreef Tim Berners-Lee vier principes van Linked Data: (i) gebruik URI's om resources te benoemen, (ii) gebruik HTTP URI's zodat die namen kunnen worden opgezocht, (iii) wanneer iemand die URI opzoekt, voorzie dan bruikbare informatie gebruikmakend van standaarden zoals Resource Description Framework, en (iv) voorzie andere URI's zodat er meer resources kunnen worden ontdekt~\cite{designissues}. Wanneer wordt voldaan aan deze vier regels, kunnen gegevens met elkaar in verbinding worden gebracht zoals hyperlinks doen voor documenten op bijvoorbeeld een webpagina.

\subsection{Uniform Resource Identifier}
URI is een begrip dat geïntroduceerd werd door de het IETF\footnote{\url{https://www.ietf.org}} als een internet standaard om een abstracte of fysische resource te identificeren aan de hand van een compacte opeenvolging van karakters~\cite{uri-ietf}. Anders dan de Internationalized Resource Identifier (IRI) kan een URI enkel bestaan uit karakters behorende tot de ASCII karakter-set\footnote{\url{https://theasciicode.com.ar/}}, terwijl deze laatste mag bestaan uit karakters van de Universal coded Character Set (Unicode/ISO 10646)~\cite{ucs}.
Een HTTP URI is een URI dat kan worden gederefereerd met een HTTP-client zodat informatie kan worden opgezocht over de resource geïdentificeerd door de URI. De mogelijkheid tot het dereferen van een URI is een kernprincipe van Linked Data: het kunnen opzoeken van iets dat je niet kent~\cite{verborgh_webfundamental}.

\begin{figure}
	\centering
	\begin{subfigure}{\textwidth}
		\centering
		\centerline{
			\includegraphics[scale=0.75]{images/rdfexample.png}
		}
	\end{subfigure}
	\caption{Voorbeeld van een \acrshort{rdf} graaf~\cite{rdfgraph_img}}
	\label{fig:rdf_example}
\end{figure}

\section{Resource Description Framework}
\label{sec:rdf}
\acrfull{rdf} is een standaard model om metadata te koppelen aan Web resources en te verwerken. Metadata zijn gegevens over gegevens, en specifiek in deze context zijn het `gegevens die Web resources beschrijven'. Dit model zorgt voor de interoperabiliteit tussen gegevens op het Web door resources te voorzien van eigenschappen en waardes behorend tot die eigenschappen. Een resources is `iets' dat beschreven wordt door een RDF-expressie. Het kan een volledig document zijn, een object in een gegevensset, het kan zelf iets zijn dat niet direct beschikbaar is op het Web bijvoorbeeld jouw fiets of een geprint boek. Een eigenschap is "een specifiek aspect, karakteristiek, attribuut, of relatie dat een resource beschrijft" \cite{rdf}.
Het linken van die metadata gebeurt in de vorm van RDF-triples: een driedelige subject-predicaat-object structuur. Deze gelinkte structuur vormt een gerichte en gelabelde graaf waarbij de takken informatie bevatten (predicaat) over de relatie tussen twee knopen (subject en object)~\cite{rdf_triple}. 
Figuur~\ref{fig:rdf_example} geeft een voorbeeld van hoe zo een graaf er kan uitzien met hieronder een bijhorende tekstuele representatie:

\begin{code}
\begin{minted}[breaklines]{turtle}
@prefix ex:  <http://example.org/> .
@prefix xsd: <https://www.w3.org/2001/XMLSchema#> .

<http://example.org/123> ex:age 43^^xsd:int .
\end{minted}
\caption{RDF-triple in turtle formaat}
\label{code:rdftriple}
\end{code}

Deze RDF-triple bestaat uit de volgende delen:
\begin{table}[h]
\begin{tabular}{|l|l|}
    \hline
    Subject (resource) & <http://example.org/123>\\
    \hline
    Predicaat (eigenschap) & ex:age\\
    \hline
    Object (literal) & 43\\
    \hline
\end{tabular}
\end{table}

\acrshort{rdf} triples kunnen geserialiseerd worden in verschillende vormen. Bovenstaande representatie maakt gebruik van de Turtle syntax\footnote{\url{https://www.w3.org/TR/turtle/}}. Andere serialisatieformaten zijn JSON-LD, RDF/XML, N-triples, ...~\cite{publishingLD}.

In dit voorbeeld wordt het RDF-subject, dat kan hier een persoon zijn, voorgesteld door een \acrshort{uri} die verwijst naar een andere bron die bijvoorbeeld namen van studenten bevat. Met deze triple wordt het subject uitgebreid met een leeftijd. Het RDF-object in figuur~\ref{fig:rdf_example} is geen \acrshort{uri} referentie maar een `RDF typed literal' gebruikt om strings, data en, in dit geval, een getal voor te stellen. Toch kunnen objecten ook verwijzen naar een \acrshort{uri} zoals in deze uitbreiding:

\begin{code}
\begin{minted}[breaklines]{turtle}
@prefix ex:   <http://example.org/> .
@prefix xsd:  <https://www.w3.org/2001/XMLSchema#> .
@prefix foaf: <http://xmlns.com/foaf/0.1/> .

<http://example.org/123> ex:age 43^^xsd:int ;
                         foaf:knows <http://example.org/456> .
\end{minted}
\caption{RDF-triple in turtle formaat -- uitbreiding van Listing \ref{code:rdftriple}}
\label{code:rdftriple_uitbreiding}
\end{code}

Aan het RDF-subject wordt nu door middel van het foaf:knows\footnote{\url{http://xmlns.com/foaf/0.1/}} predicaat een RDF-object gekoppeld dat gedefinieerd werd door een \acrshort{uri} referentie.
In deze uitbreiding werd gebruik gemaakt van een predicaatlijst. Met behulp van het `;'-teken kan het subject worden herhaald waarna predicaat en object kunnen variëren.

\section{RDF-vocabularium}
Om RDF-datamodellen op te bouwen zoals in het voorbeeld hierboven~(Listing \ref{code:rdftriple_uitbreiding}), is er een \gls{ontologie} nodig. \Glspl{ontologie} zijn op voorhand gedefinieerde concepten die relaties kunnen hebben met elkaar, en dat binnen hetzelfde \gls{ontologie} of met andere externe \glspl{ontologie} door verwijzing met een URI. Met behulp van deze documenten kan een gegevensset gemapt worden naar een \acrshort{rdf}-datamodel.
Het publiceren van een document dat een \gls{ontologie} beschrijft, gebeurt opnieuw als door de principes van Linked Data te volgen. De term `ontologie' wordt ook gebruikt om een RDF-vocabularium te benoemen, het duidt echter op een meer complexe collectie van termen. `Vocabularium' wordt in een lossere omgeving toegepast. Alhoewel de twee termen van elkaar verschillen is er geen duidelijk onderscheid over wanneer een dergelijk document een vocabularium of ontologie wordt genoemd~\cite{ontology}.

\subsection{RDF Schema}
RDF Schema (RDFS) is het basis RDF-vocabularium waarmee andere vocabularia kunnen worden gemodelleerd. Het definieert klasses, eigenschappen en datatypes die de basis blokken vormen om andere vocabularia te definiëren~\cite{verborgh_webfundamental}. Het schema definieert concepten in twee namespaces: \url{https://www.w3.org/1999/02/22-rdf-syntax-ns#} en \url{https://www.w3.org/2000/01/rdf-schema#} met respectievelijk de alom aanvaarde prefixen `rdf' en `rdfs'.
Veel gebruikte termen zijn:
\textit{rdfs:Resource} wat een klasse is waarvan alles een instantie is. Bijkomende termen kunnen die instanties verder specialiseren.
\textit{rdfs:Class} is een klasse voor het definiëren van resources die een verzameling van zaken voorstelt.
\textit{rdf:Property} is een klasse voor resources die als predicaat in een RDF-triple kunnen worden gebruikt.
\textit{rdfs:label} dat een eigenschap is dat een voor mensen leesbare naam geeft aan een resource.
\textit{rdf:type} is een eigenschap dat aanduidt dat de resource een instantie is van een klasse.
\textit{rdfs:domain} is een eigenschap dat de klasse van mogelijke subjecten van een eigenschap aantoont. In Listing \ref{code:rdf_example} is \textit{foaf:knows} een eigenschap dat enkel op een instantie van de klasse \textit{foaf:Person} kan worden toegepast.
\textit{rdfs:range} is een eigenschap dat de klasse van mogelijke objecten van een eigenschap aan toont~\cite{verborgh_webfundamental}. In Listing \ref{code:rdf_example} kan het object van de \textit{foaf:knows} eigenschap enkel een object zijn van de klasse \textit{foaf:Person}.

Het codevoorbeeld in Listing~\ref{code:rdf_example} omschrijft het \textit{foaf:knows} predicaat uit Listing~\ref{code:rdftriple_uitbreiding}. Bij het definiëren van dit concept wordt gebruik gemaakt van andere \glspl{ontologie} (hier rdf, rdfs en owl) door via \acrshort{uri}s te verwijzen naar concepten uit die \glspl{ontologie}. Door het concept `knows' te interpreteren wordt duidelijk dat het om een eigenschap gaat dat moet gebruikt worden binnen het domein \textit{foaf:Person}. Het object dat de eigenschap aanvaardt, wordt aangeduid met de `range' eigenschap en is in dit voorbeeld ook de klasse \textit{foaf:Person}.

\subsection{Ontology Web Language}
Terwijl RDFS vooral de basis concepten voor relaties definieert, kan het Ontology Web Language (OWL) vocabularium gezien worden als een uitbreiding van RDFS doordat het meer geavanceerde concepten definieert.
OWL wordt gebruikt om klasses, eigenschappen en individuen verder te specificeren~\cite{verborgh_webfundamental}.
De termen `instantie' en `individu' hebben hier elk een andere betekenis. Een klasse gedefinieerd door ontwikkelaars is een \textit{instantie} van de bestaande \textit{owl:Class} klasse. Terwijl een een \textit{individu} een instantie is van de (nieuwe) gedefinieerde klasse afgeleid van \textit{owl:Class}~\cite{owl}.
In Listing~\ref{code:rdf_example} zijn de klassen \textit{ex:Dier} en \textit{ex:Kat} instanties van \textit{owl:Class} en is \textit{ex:sam01} een individu van de klasse \textit{ex:Kat}.

\begin{code}
\begin{minted}[breaklines]{turtle}
@prefix ex:  <https://example.org/rdf#>
@prefix owl: <https://www.w3.org/2002/07/owl#> .
@prefix rdf: <http://www.w3.org/1999/02/22-rdf-syntax-ns#> .
@prefix rdfs: <http://www.w3.org/2000/01/rdf-schema#> .
@prefix foaf: <http://xmlns.com/foaf/0.1/> .

ex:Dier a owl:Class ;
        rdfs:label "dier" ;
        rdfs:comment "Een levend wezen dat geen mens en plant is." .
        
ex:Kat  a ex:Dier ;
        rdfs:label "kat" ;
        rdfs:comment "Een gewerveld dier behorend tot de groep van katachtigen" .
        
ex:sam01 a ex:Kat ;
         foaf:name "Sam" .

foaf:knows a rdf:Property,
             owl:ObjectProperty ;
            rdfs:comment "A person known by this person (indicating some level of reciprocated interaction between the parties)." ;
            rdfs:domain foaf:Person ;
            rdfs:isDefinedBy foaf: ;
            rdfs:label "knows" ;
            rdfs:range foaf:Person .
\end{minted}
\caption{Voorbeeld van een zelf gedefinieerde instantie en individu en een instantie van het FoaF RDF-vocabularium}
\label{code:rdf_example}
\end{code}

\section{SPARQL}
SPARQL is een query-taal om gegevens uit een RDF-gegevensset op te vragen door een `basic graph pattern' (BGP) op te bouwen. Zo een BGP bestaat uit een collectie van triplet-patronen die lijken op de RDF-tripletten uit sectie~\ref{sec:rdf}. Het verschil is hier dat de URI van het subject, predicaat of object kan worden vervangen door een variabele. Er is een match tussen de BGP en een subgraaf van de gegevensset als RDF-termen kunnen gesubstitueerd worden voor de variabelen uit de GBP. Het resultaat is een RDF-graaf gelijkaardig aan die van de subgraaf~\cite{sparql}. Een simpel voorbeeld met als gegevensset de RDF-grafen uit Listing~\ref{code:rdf_example} is die  in Listing~\ref{code:sparql} met resultaat `name: "Sam"'.

\begin{code}
\begin{minted}[breaklines]{sparql}
@prefix ex:  <https://example.org/rdf#>
@prefix foaf: <http://xmlns.com/foaf/0.1/> .

SELECT ?name 
WHERE{
    ?s a ex:Kat .
    ?s foaf:name ?name .
}

\end{minted}
\caption{Voorbeeld SPARQL-query met als datasource Listing~\ref{code:rdf_example}}
\label{code:sparql}
\end{code}

SPARQL heeft vier query vormen die een resultaat teruggeven. (i) `SELECT' geeft de variabelen uit de BGP terug die konden worden gesubstitueerd met RDF-termen in de graaf uit de RDF-gegevensset. (ii) De `CONSTRUCT'-clausule creëert tripletten die aan de condities van de clausule voldoen, (iii) terwijl de `ASK'-clausule checkt of de gegevens die worden gequeryd wel bestaan. (iv) De `DESCRIBE'-clausule geeft informatie terug over een resource~\cite{verborgh_webfundamental}.

\section{Vijfsterrenmodel}
\label{sec:linked_open_data}
Om mensen te stimuleren kwalitatief goede Linked Data te publiceren, ontwikkelde Tim Berners-Lee een vijfsterrenmodel. 
Linked Open Data krijgt vijf sterren wanneer het voldoet aan een aantal voorwaarden: (i) de gegevens moeten online onder een open licentie beschikbaar zijn, (ii) de gegevens moeten machine-leesbaar zijn en mogen dus géén scan van een document zijn, (iii) daar bovenop moet het in een open bestandsformaat beschikbaar zijn (bijvoorbeeld csv in plaats van excel), (iv) bovendien moet het gebruikmaken van de w3c\footnote{\url{https://www.w3.org/standards/}} standaard bv. \acrfull{rdf} zodat anderen naar de dataobjecten kunnen verwijzen. (v) De gegevens zelf moeten ook verwijzen naar gegevens van anderen zodat er meer context kan gecreeërd worden.
In de inleiding van dit hoofdstuk werd duidelijk dat overheden en bedrijven meer gegevens onder een open licentie moeten delen. Dit geldt natuurlijk niet voor belangrijke gegevens die intern moeten blijven of voor persoonlijke gegevens. Linked Data kan even goed vijf sterren hebben zonder dat het onder een open licentie moet worden gepubliceerd. Indien er echter over Linked Open Data gesproken wordt, moet het uiteraard wel `open' zijn om sterren te kunnen verdienen.

\section{Probleemstelling en doel van de masterproef}
\label{sec:problem}
De non-interoperabiliteit van gegevens van de publieke sector diensten resulteert in een moeilijke samenwerking tussen partijen die gegevens van elkaar gebruiken. 
Het herpubliceren van die gegevens als Linked Open Data en hergebruiken van identificatoren kan de interoperabiliteit tussen gegevens verhogen.
Meer specifiek moeten de gegevens interoperabel zijn met de andere gegevens van de Vlaamse overheid. 
Hieraan wordt, op dit moment van schrijven, in Vlaanderen al volop gewerkt met het Vlaams interoperabiliteitsprogramma `Open Standaarden voor Linkende Organisaties' of kortweg: OSLO\footnote{https://data.vlaanderen.be/}. 
Dit programma ontwikkeld vocabularia om gegevens van (Belgische) overheden en bedrijven te publiceren als \acrshort{lod}. 
De doelstelling van deze masterproef is om ook gegevens van \glspl{mobop} mee in het OSLO-traject te krijgen door een RDF-vocabularium te voorzien waarin de gegevens OSLO-conform kunnen worden gepubliceerd en methodes aan te reiken bestaande gegevens te mappen op dat nieuwe vocabularium.
Deze twee stappen vormen de `oprit' de OSLO-pijplijn en wordt verder in deze scriptie de `on-boarding' van gegevens genoemd. Na deze on-boarding fase kunnen de gegevens verder hun weg vinden doorheen de pijplijn naar onder andere archivering en kunnen 
queries opgebouwd worden die meerdere thematisch verschillende gegevenssets aanspreken.

In deze masterproef zullen de gegevens van \glspl{fietsdeelop} Blue Bike en Velo als aanknopingspunt worden onderzocht. Als eerste contributie wordt bekeken of er standaard specificaties zijn waaruit concepten en structuren kunnen worden hergebruikt in een nieuw RDF-vocabularium dat zal worden ontwikkeld. 
Dat vocabularium moet een connectie hebben met de bestaande OSLO-vocabularia zodat de gegevens interoperabel zijn met andere gegevens van de Vlaamse overheid. Om die connectie te maken ga ik opzoek naar een OSLO-vocabularium met concepten die aansluiten bij de termen die reeds in gegevenssets rond fietsdeelsystemen worden gebruikt.
Ter demonstratie worden gegevens met behulp van het ontwikkelde vocabularium gemapt naar een \acrshort{rdf}-gegevensset en gepubliceerd als \acrlong{lod}. Tijdens het onderzoek en de implementatie moet er zo generiek mogelijk te werk worden gegaan zodat in een volgende fase andere types vervoermiddelen kunnen worden geïmplementeerd. Als laatste contributie moeten er oplossingen worden aangeboden aan de operatoren om hun gegevens te herpubliceren gebruikmakend van het ontwikkelde vocabularium zonder dat dit een extra last brengt bij de publicatie van de gegevens.

\chapter{State of the art}

Er zijn specificaties (GBFS\footnote{\url{https://github.com/NABSA/gbfs}}, GTFS\footnote{\url{https://developers.google.com/transit/gtfs/reference}}, MDS\footnote{\url{https://github.com/openmobilityfoundation/mobility-data-specification/tree/master}}) die een zo generiek mogelijk gegevensmodel voorstellen zodat \glspl{mobop} hun gegevens op een interoperabele manier kunnen publiceren. Helaas zijn deze specificaties volgens het sterrenschema van Tim Berners-Lee geen vijf sterren waard. De gepubliceerde gegevens voldoen daarom niet aan de concepten van het European Interoperability Framework door tekortkomingen beschreven in secties~\ref{sec:GBFS} en \ref{sec:ngsi}.

Daarnaast bestaan er al \glspl{ontologie} (Mobivoc\footnote{\url{http://schema.mobivoc.org}} en OSLO mobiliteit: trips \& aanbod\footnote{\url{https://data.vlaanderen.be/doc/applicatieprofiel/mobiliteit-trips-en-aanbod}}) waarmee gegevens van \glspl{mobop} als \acrshort{lod} kunnen worden gepubliceerd. Deze vocabularia hebben beperkingen die omschreven worden in secties~\ref{sec:trips&aanbod} en \ref{sec:mobivoc} waardoor ze geen volledige oplossing bieden voor fietsdeeloperator gerelateerde gegevens. Wel kunnen deze \glspl{ontologie} gebruikt worden als basis voor een nieuwe ontologie of kan er een uitbreiding voor gemaakt worden. 

\section{General Bikeshare Feed Specification}
\label{sec:GBFS}
GBFS is een open standaard voor deelmobiliteit dat in een uniform formaat real-time gegevens publiceert. Het voordeel aan deze specificatie is dat het om een open standaard gaat en bruikbaar is door iedereen en voor verschillende toepassingen. Tussen systemen die gebruikmaken van GBFS is er interoperabiliteit, maar daarbuiten niet. Wie met de standaard werkt, wordt verplicht de gegevensset te publiceren in het JSON-gegevensformaat. Hierdoor blijven semantische discussies over de betekenis van de gebruikte sleutels in het model mogelijk. Dit veroorzaakt een gebrek aan interoperabiliteit bij gegevensoverdracht tussen systemen buiten de grenzen van GBFS, zoals diensten van overheden. Zonder mapping kunnen andere systemen niet aansluiten op deze gegevens gepubliceerd in de open standaard. Daarnaast vereist de standaard geen herbruikbare identificatoren waardoor de gegevens van objecten niet uitbreidbaar zijn. 

Dankzij het gebruikte JSON-formaat van GBFS zijn gegevens die onder een open licentie worden gepubliceerd drie sterren  (\ast \ast \ast) waard volgens het vijfsterrenmodel.
Er kan hier echter niet gesproken worden over Linked Data gezien
er geen gebruik wordt gemaakt van open standaarden van w3c (RDF) en de gegevens niet op te vragen zijn met SPARQL. 

Toch lijkt het model van deze specificatie een goed startpunt te zijn om gegevens van fietsdeeloperatoren in te publiceren. GBFS bewees zichzelf reeds doordat het, op het moment van schrijven,  door verschillende operatoren wereldwijd op 452 plaatsen wordt gebruikt~\cite{GBFS_systems}. Operatoren die gebruikmaken van de standaard hier in België zijn: Donkey Republic (actief in Gent) en Bird (actief in Antwerpen). De lijst~\footnote{\url{https://github.com/NABSA/gbfs/blob/master/systems.csv}} bevat voornamenlijk operatoren die deelfietsen aanbieden. Toch zijn er ook operatoren die andere voertuigen aanbieden volgens hetzelfde \gls{deeleconomie} principe en met hetzelfde gegevensmodel. Vervoermiddel-specifieke gegevens zoals het aantal beschikbare voertuigen en docks van een bepaald type, worden niet rechtstreeks met een eigenschap toegevoegd aan een stations-object. In plaats daarvan heeft een stations-object een array met objecten die zich van elkaar onderscheiden door de toevoeging van een eigenschap met het type vervoermiddel waartoe deze informatie hoort.
De GBFS-standaard voorziet objecten met eigenschappen voor informatie over stations en hun real-time status, voertuigen en hun status en andere systeem eigen informatie~\footnote{\url{https://github.com/NABSA/gbfs/blob/master/gbfs.md}}.

\section{NGSI}
\label{sec:ngsi}
NGSI\footnote{\url{https://fiware-ges.github.io/orion/api/v2/stable/}} is een verzameling van gegevensmodellen voor uiteenlopende toepassingen: Smart Cities, Smart Agrifood, Smart Environment, Smart Sensoring, Smart Enery, Smart Water en andere. Een van de gegevensmodellen dat FIWARE, de organisatie achter de specificatie, ontwikkeld heeft is `Bike Hire Docking Station'. Net zoals GBFS kan dit gegevensmodel worden gebruikt door fietsdeeloperatoren om hun gegevens te publiceren in JSON-formaat. Het verschil met GBFS is dat iedere eigenschap nog extra eigenschappen heeft. Die `sub-eigenschappen' houden metadata bij voor hun eigenschap, zoals een timestamp. Eén van die sub-eigenschappen, misschien wel de belangrijkste, is de `value'-eigenschap die de waarde van zijn super-eigenschap bijhoudt. Verdere vergelijkingen tussen NGSI en GBFS vallen buiten de scope van deze scriptie.

Deze specificatie kan met dezelfde argumenten als die voor GBFS worden beoordeeld met drie sterren (\ast \ast \ast) volgens het vijfsterrenmodel.

\section{NGSI-LD}
\label{sec:ngsi-ld}
NGSI-LD is gebasseerd op JSON-LD\footnote{\url{https://json-ld.org/}} en is de Linked Data versie van de klassieke NGSI gegevensmodellen. JSON-LD kan door middel van het toevoegen van een `@context'-object een JSON gegevensset beschikbaar maken als Linked Data, querybaar met een SPARQL query.
Een NGSI-LD gegevensmodel zit anders in elkaar dan een klassiek RDF-model door het invoeren van sub-eigenschappen. Daarom introduceert FIWARE de NGSI-LD core context\footnote{\url{https://uri.etsi.org/ngsi-ld/v1/ngsi-ld-core-context.jsonld}} om termen te mappen op URI's. In de gegevensmodellen context\footnote{\url{https://fiware.github.io/data-models/context.jsonld}} van NGSI, worden URI's beschreven om de termen die eigen zijn aan de NGSI modellen te mappen op URI's waardoor er binnen het NGSI-LD ecosysteem interoperabiliteit is. De URI's zijn echter geen HTTP URI's, het is niet mogelijk om bijvoorbeeld de URI \url{https://uri.fiware.org/ns/data-models#availableBikeNumber} te derefereren. Volgens de vier principes van Linked Data (sectie \ref{semantisch_web}) is dit wel een vereiste waardoor het LD gedeelte in NGSI\textbf{-LD} niet op zijn plaats staat. NGSI-LD kan dus geen Linked Data genoemd worden omdat de URI's geen bruikbare informatie leveren over de predicaten.

\section{OSLO mobiliteit: trips \& aanbod}
\label{sec:trips&aanbod}
De OSLO mobiliteit: trips \& aanbod specificatie is onderdeel van het OSLO programma beschreven in sectie~\ref{sec:OSLO}. Gezien de aard van OSLO kan een gegevensset gepubliceerd volgens dit RDF-vocabularium worden beoordeeld met vijf sterren (\ast \ast \ast \ast \ast). Het vocabularium voorziet allerhande termen voor gegevensuitwisseling over door personen uitgevoerde reizen en de mobiliteitsdiensten die ze daarvoor ter beschikking hebben~\cite{oslomobiliteit}. Dit lijkt een goed kandidaat-gegevensmodel waarmee gegevens van fietsdeeloperatoren kunnen worden gepubliceerd. De klassen en eigenschappen beschreven door het vocabularium zijn echter beperkt. De RDF-termen beschrijven hoofdzakelijk zeer low level concepten en begrippen zoals aanbieder, mobiliteitsdienst, prijsplan, reiziger, route enz...~\footnote{\url{https://data.vlaanderen.be/doc/applicatieprofiel/mobiliteit-trips-en-aanbod\#overview}} Om gegevens van fietsdeeloperatoren te publiceren is er meer nood aan RDF-termen gelijkaardig aan de properties van GBFS.

\section{MobiVoc: Open Mobility Vocabulary}
\label{sec:mobivoc}
Deze specificatie maakt ook gebruik van RDF. Gegevens gepubliceerd met dit RDF-vocabularium hebben potentieel om vijf sterren (\ast \ast \ast \ast \ast) waard te zijn. Mobivoc profileert zich als een vocabularium voor toekomstgerichte mobiliteit. Hiermee lijkt ook dit vocabularium een goede kandidaat voor het publiceren van gegevens van fietsdeeloperatoren als LOD. Het vocabularium bevat een ruim assortiment aan RDF-termen zoals `bicycle parking station' dat een subklasse is van `parking facility'. Daarnaast bevat het ook termen om eigenschappen van voertuigen en stations te modelleren zoals de real-time capaciteit. 

Hetgeen dat hier mist is een connectie met OSLO zodat er kan gelinkt worden naar reeds bestaande gegevenssets in Vlaanderen. Op die manier kunnen nieuwe en bestaande fietsenstallingen voor deelfietsen gelinkt worden aan adressen, plaatsen of openbare gebouwen die reeds als LOD objecten ter beschikking worden gesteld met de OSLO-vocabularia. 
\chapter{Een RDF ontologie voor gegevens van  fietsdeeloperatoren}
\label{chap:ontologie_voor_fietsdeeloperatoren}
\Glspl{mobop} publiceren hun gegevens in verschillende gegevensmodellen en -formaten waardoor die gegevenssets niet interoperabel zijn met andere gegevens van de Vlaamse overheid. Het OSLO mobiliteit: trips \& aanbod vocabularium zou een uitbreiding moeten krijgen zodat we gegevens van Vlaamse fietsdeeloperatoren in het OSLO programma kunnen krijgen. GBFS biedt de nodige object eigenschappen en structuur die nodig zijn voor een model voor gegevens rond deelmobiliteit. 

De secties die volgen beschrijven hoe de uitbreiding op OSLO mobiliteit: trips \& aanbod als proof of concept tot stand zijn gekomen. De uitbreiding is niet volledig, maar generiek zodat er aan kan worden verder gebouwd indien het vocabularium wordt opgenomen in OSLO.
In tabel \ref{tab:oslo_prefixes} staan enkele OSLO-prefixes waarnaar zal worden verwezen in onderstaande secties. De technische informatie van iedere RDF-klasse waarnaar wordt gerefereerd kan worden teruggevonden door de URI te volgen gevormd door de prefix geconcateneerd met de RDF-term.

\begin{table}[]
\centering
\caption{gebruikte prefixes}
\label{tab:oslo_prefixes}
\begin{tabular}{ll}
ext:     & \url{https://stijnbrysbaert.github.io/OSLO-extension/vocabulary.ttl#} \\
dct   & \url{http://purl.org/dc/terms/} \\
mob   & \url{https://data.vlaanderen.be/ns/mobiliteit-trips-en-aanbod#} \\
net   & \url{https://data.vlaanderen.be/ns/netwerk#}         \\
tn    & \url{https://data.vlaanderen.be/ns/transportnetwerk#} \\
rdf   & \url{https://www.w3.org/1999/02/22-rdf-syntax-ns#} \\
weg   & \url{https://data.vlaanderen.be/ns/weg#}
\end{tabular}
\end{table}

\section{Station}
Allereerst is er nood aan een `Station'-klasse dat instanties van stations omschrijft. Een station in deze context is een object in het mobiliteitslandschap waar vervoermiddelen kunnen geparkeerd en gebruikt worden. Afgeleid uit het GBFS station status gegevensmodel\footnote{\url{https://github.com/NABSA/gbfs/blob/master/gbfs.md\#station_statusjson}} kan een station de real-time beschikbaarheid van voertuigen weergeven. In de GBFS specificatie wordt de capaciteit en beschikbaarheid van voertuigen gemodelleerd zodat het mogelijk is die eigenschappen bij te houden voor meerdere verschillende soorten vervoermiddelen: (elektrische)fiets, auto, step, ... Op die manier kan de gegevensset van een station worden aangevuld wanneer er een nieuw type voertuig wordt in ondergebracht.

tn:Transportpunt kan als superklasse worden gebruikt voor deze nieuwe klasse om een connectie te voorzien naar OSLO. Deze superklasse is zelf een subklasse van tn:Transportobject en tn:Netwerkelement en kan worden gelinkt aan een geografisch punt met het tn:geometrie predicaat.

\begin{table}[h]
\begin{tabular}{|l|l|}
\hline
\textbf{type}     & \textbf{klasse}                                                                                \\ \hline
URI               & {\color[HTML]{000000} ext:Station} \\ \hline
Specialisatie van & \begin{tabular}[c]{@{}l@{}}\url{https://data.vlaanderen.be/ns/transportnetwerk#Transportpunt}\\ \end{tabular} \\ \hline
Definitie         & Station waar voertuigen kunnen worden geplaatst en gebruikt.                                   \\ \hline
\end{tabular}
\end{table}

\section{VoertuigenBeschikbaar}
Het idee uit de GBFS specificatie om gegevens van verschillende vervoersmiddelen in eenzelfde station bij te houden, nemen we over naar deze OSLO uitbreiding. De klasse waarin het aantal beschikbare voertuigen, bijvoorbeeld deelfietsen, wordt gemodelleerd, moet generiek zijn. De VoertuigenBeschikbaar-klasse zal voor een specifiek vervoermiddel bijhouden hoeveel items er op dat moment beschikbaar zijn om te gebruiken. In dit domein kunnen twee eigenschappen worden aanvaard: aantal en dct:type. Van deze klasse kunnen er meerdere instanties gelinkt worden aan de station instantie.

\begin{table}[h]
\begin{tabular}{|l|l|}
\hline
\textbf{type}     & \textbf{klasse}                                                                                \\ \hline
URI               & {\color[HTML]{000000} ext:VoertuigenBeschikbaar} \\ \hline
Definitie         & Het aantal voertuigen van een bepaalt type dat beschikbaar is.                                   \\ \hline
\end{tabular}
\end{table}

\subsection{aantal}
Deze eigenschap duidt het aantal voertuigen van een bepaald vervoermiddel aan.

\begin{table}[h]
\begin{tabular}{|l|l|}
\hline
\textbf{type}     & \textbf{eigenschap}                                                                                \\ \hline
URI               & {\color[HTML]{000000} ext:aantal} \\ \hline
Definitie         & Aantal items van een instantie  \\ \hline
Domein & \begin{tabular}[c]{@{}l@{}}ext:VoertuigenBeschikbaar\\ext:VoertuigenDocks\end{tabular} \\ \hline
Bereik & \url{http://www.w3.org/2001/XMLSchema#integer} \\ \hline
\end{tabular}
\end{table}

\subsection{type}
De dct:type eigenschap verwacht volgens het applicatieprofiel van OSLO mobiliteit: trips \& aanbod een instantie van het rdf type Resourcetype. Resourcetype werd echter nog niet gedefinieerd in het applicatieprofiel. Het omschrijven van deze klasse valt buiten de scope van deze scriptie. 

% In het wegen-vocabularium\footnote{\url{https://data.vlaanderen.be/ns/weg}} bestaat er een eigenschap weg:voertuigtype dat ook ideaal lijkt om het type voertuig aan te duiden. Deze eigenschap kan helaas enkel worden toegepast worden in het weg:Rijrichting domein. Er zal een weloverwogen beslissing moeten worden gemaakt over welke RDF term zal worden 

\section{DocksBeschikbaar}
De klasse `DocksBeschikbaar' wordt gelinkt met dezelfde eigenschappen als de VoertuigenBeschikbaar-klasse: aantal en type. De klasse houdt het aantal \glspl{dock} dat beschikbaar is in een station voor een bepaalt vervoermiddel bij.

\section{Beschikbaarheid}
De beschikbaarheid-klasse gedefinieerd in OSLO mobiliteit: trips \& aanbod kan gelinkt worden aan een station en bevat instanties die betrekking hebben tot de mate waarin iets voorhanden is voor gebruik. Een station heeft exact één instantie van de beschikbaarheid-klasse en wordt met behulp van de mob:Transportobject.beschikbaarheid eigenschap gelinkt. Gezien een station een subklasse is van mob:Transportpunt en die klasse op zijn beurt weer een subklasse is van mob:Transportobject, wordt de mob:Transportobject.beschikbaarheid eigenschap binnen het juiste domein gebruikt.

Wat volgt zijn twee eigenschappen toepasbaar op de mob:Beschikbaarheid-klasse waarmee gegevens rond de beschikbaarheid van voertuigen en docks in een station kunnen worden gelinkt.

\subsection{voertuigTypesBeschikbaar}
Deze eigenschap vormt de relatie tussen de beschikbaarheids-instantie van een station met het real-time aantal beschikbare voertuigen van een bepaalt type.

\begin{table}[h]
\begin{tabular}{|l|l|}
\hline
\textbf{type}     & \textbf{eigenschap} \\ \hline
URI               & ext:voertuigTypesBeschikbaar \\ \hline
Definitie         & Het aantal voertuigen van een bepaalt type dat beschikbaar is.                                   \\ \hline
Domein & mob:Beschikbaarheid \\ \hline
Bereik & ext:VoertuigenBeschikbaar \\ \hline
\end{tabular}
\end{table}

\subsection{voertuigDocksBeschikbaar}
Deze eigenschap vormt de relatie tussen de beschikbaarheids-instantie van een station met het real-time aantal beschikbare docks voor een bepaalt type voertuig. 

\begin{table}[h]
\begin{tabular}{|l|l|}
\hline
\textbf{type}     & \textbf{eigenschap} \\ \hline
URI               & ext:voertuigDocksBeschikbaar \\ \hline
Definitie         & Beschikbaarheid van het aantal plaatsen specifiek voor een bepaalt type voertuig.  \\ \hline
Domein & mob:Beschikbaarheid \\ \hline
Bereik & ext:DocksBeschikbaar \\ \hline
\end{tabular}
\end{table}

\section{Voorbeeld}
Figuur~\ref{fig:rdf_gegevensset_voorbeeld} beeldt de graaf af die een mogelijke gegevensset, gepubliceerd volgens de OSLO uitbreiding, visualiseert. Dankzij de klasses VoertuigTypesBeschikbaar en DocksBeschikbaar kunnen er onbeperkt aantal instanties gelinkt worden die per type voertuig informatie bevat.

\begin{figure}[h]
	\centering
	\begin{subfigure}{\textwidth}
		\centering
		\centerline{
			\includegraphics[width=\textwidth]{images/rdf_dataset.pdf}
		}
	\end{subfigure}
	\caption{Gegevensset gebruikmakend van RDF model}
	\label{fig:rdf_gegevensset_voorbeeld}
\end{figure}
\chapter{Gegevens on-boarding}
\label{chap:on-boarding}

Het is moeilijk een gegevensmodel en -formaat te verplichten aan \glspl{mobop}. Het zou echter ideaal zijn mochten eigenaren van gegevens zelf hun gegevens in het gepaste RDF-model publiceren. Uit ervaring blijkt dat operatoren die deze gegevens produceren en publiceren geen middelen (tijd en geld) hebben of hierin willen investeren. De oplossing zit daarom niet in het overtuigen van die operatoren een bepaalde procedure te volgen bij het publiceren van hun gegevens. De Vlaamse overheid zal moeten investeren in een hulpmiddel om bestaande gegevens van operatoren te mappen naar een \acrshort{rdf}-model zodat het beschikbaar wordt als \acrshort{lod}~\cite{vernaillen}.

Om gegevens rond \glspl{mobop} te linken met andere gegevens in Vlaanderen, werd de OSLO mobiliteit: trips \& aanbod ontologie uitgebreid (hoofdstuk ~\ref{chap:ontologie_voor_fietsdeeloperatoren}). Eens gegevens gepubliceerd worden conform die OSLO uitbreiding, zit het in de OSLO pijplijn. Deze pijplijn kan toegang bieden tot allerhande services die verdere bewerkingen uitvoeren op gegevens: archivering, bewaren en ophalen van historische gegevens, ... De pijplijn en eraan gekoppelde services zijn nog volop in ontwikkeling. Eens ze beschikbaar zijn, zijn ook de gegevens klaar om er gebruik van te maken. Deze scriptie en specifiek deze sectie beschrijft deze on-boarding: het mappen van niet OSLO conforme gegevens naar het wel OSLO conform zijn.

In volgende secties zullen gegevenssets gepubliceerd met het GBFS en NGSI model gebruikt worden aangezien dit populaire specificaties zijn bij fietsdeeloperatoren.

\section{RML.io}
\label{sec:rml}
JSON is een populair formaat voor het bewaren en versturen van gegevens. Het wordt meestal gebruikt als formaat om gegevens, opgevraagd via een API, terug te geven. Veel gebruikte gegevensmodellen om gegevens rond fietsdeeloperatoren te publiceren in JSON zijn GBFS en NGSI. GBFS werd reeds omschreven in sectie~\ref{sec:GBFS}. Voorbeelden zijn de velo.json (bijlage A) dat gebruikmaakt van NGSI. En donkey.json (bijlage B) dat gebruikmaakt van GBFS. 
Een eerste use case om van deze gegevens Linked Data te genereren is door middel van de RML toolchain\footnote{\url{https://rml.io/}}. De eerste stap is het opbouwen van RML regels als tussenstap voor het genereren van LD. RML kan met YARRRML gegenereerd worden. Het codevoorbeeld in bijlage C is een YARRRML document waarmee de velo.json gegevensset uit bijlage A naar RML kan worden geserialiseerd.
Daarna wordt door middel van een RMLMapper\footnote{\url{https://github.com/RMLio/rmlmapper-java}} uit de RML regels Linked Data gegenereerd\footnote{\url{https://github.com/stijnbrysbaert/mapper}}.

\section{JSON naar JSON-LD}
\label{sec:json2jsonld}
Als tweede use case kan een JSON gegevensset omgevormd worden naar een JSON-LD formaat. JSON-LD brengt de mogelijkheid naar JSON om objecten aan elkaar te linken en te refereren naar andere objecten buiten het document met behulp van URIs (sectie ~\ref{sec:ngsi-ld}). 
Er zijn enkele mogelijkheden om de @context op te bouwen die dit mogelijk maakt. Om te beginnen kunnen de sleutels van het JSON-object gemapt worden op voorlopige dummy URIs. Om te beginnen kan bijvoorbeeld de dummy URI \url{http://example.org/rdf#} als prefix worden gebruikt zoals in voorbeeld 1 (\ref{subsec:vb1}). Dit zorgt natuurlijk niet voor interoperabiliteit met andere gegevenssets, maar het geeft een idee van hoe de @context werkt.
Als tweede stap is het mogelijk om bij simpele RDF modellen de URIs te gebruiken van dat RDF model. Met simpel wordt hier bedoeld dat een instantie van een RDF-klasse enkel relaties heeft van het subject naar een literal. Wanneer er instanties zijn die referenties hebben naar andere instanties van klasses, wordt het moeilijk dit zonder meer dan met een @context op te lossen.

Wanneer de JSON bestaat uit één enkel object, kan de @context als extra object worden toegevoegd. Wanneer het om een array van objecten gaat, moeten die nodes gegroepeerd worden door middel van het `@included'-keyword. De array van objecten wordt toegevoegd aan een root-object, waartoe ook de @context behoort, met de sleutel @included.

\subsection{Voorbeeld 1}
\label{subsec:vb1}
De @context in codevoorbeeld~\ref{code:context} kan worden gebruikt om het JSON-object in bijlage E te transformeren naar JSON-LD. Het is belangrijk de objecten met sleutels `id' en `type' toe te voegen aan de context om het geheel te laten werken. Het mapt de betreffende sleutels op de JSON-LD sleutelwoorden `@id' en `@type'. Op die manier zullen de sleutels worden geïnterpreteerd als respectievelijk het subject en het rdf:type object van de gegevensset.

\begin{code}
\begin{minted}[breaklines]{json}
{
    "@context": [
        {
            "ext": "https://stijnbrysbaert.github.io/OSLO-extension/vocabulary.ttl#"
        },
        {
            "example": "http://www.example.org/rdf#"
        },
        {
            "availableBikeNumber": {
                "@id": "example:availableBikeNumber"
            }
        },
        {
            "freeSlotNumber": {
                "@id": "example:freeSlotNumber"
            }
        },
        {
            "BikeHireDockingStation": {
                "@id": "ext:Station"
            }
        },
        {
            "id": "@id"
        },
        {
            "type": "@type"
        }
    ]
}
\end{minted}
\caption{@context met dummy en werkelijke URIs}
\label{code:context}
\end{code}

In deze context wordt als voorbeeld de sleutel ´BikeHireDockingStation' gemapt op ext:Station. Hiermee wordt aangetoond dat het mogelijk is ook URIs van eigen RDF-modellen te mappen op de sleutels van objecten. Simpele gegevenssets kunnen zo met enkel en alleen een @context gemapt worden naar een simpel RDF-model. 

\subsection{Voorbeeld 2}
\label{subsec:vb2}
Om de JSON in bijlage D naar JSON-LD te transformeren is een extra @context nodig bovenop de context in \ref{code:context}. Deze JSON gegevensset maakt gebruik van een van de gegevensmodellen van de NGSI specificatie (Bike Hire Docking Station) en werd gepubliceerd in genormaliseerde vorm (\ref{sec:ngsi-ld}). Daarom is het belangrijk om de externe context \url{https://uri.etsi.org/ngsi-ld/v1/ngsi-ld-core-context.jsonld} toe te voegen. Deze zal de sub-eigenschappen mappen op een URI zodat ze te queryen zijn. Zo zal bijvoorbeeld de sub-eigenschap `value' gemapt worden op URI \url{https://uri.etsi.org/ngsi-ld/hasValue}.
Om problemen te vermijden bij het overschrijven van URIs die zijn opgenomen in de ngsi-ld core context, zet je deze context best bovenaan in het context-object.

Gezien het NGSI gegevensmodel zal een CONSTRUCT-query nodig zijn om de sub-eigenschappen weg te werken. Deze CONSTRUCT-query zal RDF-triples genereren met de nodes die het in de WHERE-clausule van de SPARQL query uit de gegevensbron haalt. Het resultaat is een gegevensset gemapt naar het gewenste RDF model, in dit voorbeeld de uitbreiding op OSLO mobiliteit: trips \& aanbod. 

\begin{code}
\begin{minted}[breaklines]{sparql}
PREFIX dct: <http://purl.org/dc/terms/>
PREFIX example: <http://www.example.org/rdf#>
PREFIX ext: <https://stijnbrysbaert.github.io/OSLO-extension/vocabulary.ttl#>
PREFIX ngsild: <https://uri.etsi.org/ngsi-ld/>
PREFIX trips: <https://data.vlaanderen.be/ns/mobiliteit/trips-en-aanbod#>

CONSTRUCT{
    ?s a ext:Station .
    ?s trips:Transportobject.beschikbaarheid _:b .
    _:b ext:voertuigTypesBeschikbaar _:items .
    _:items a ext:VoertuigenBeschikbaar .
    _:items ext:aantal ?aantal_bike .
    _:b ext:voertuigDocksBeschikbaar _:docks .
    _:docks a ext:DocksBeschikbaar .
    _:docks ext:aantal ?aantal_slot .
    _:items dct:type _:type .
    _:type a trips:Resourcetype .
    _:docks dct:type _:type .
}
WHERE {
    ?s a ext:Station .
    ?s example:availableBikeNumber ?obj_abn .
    ?obj_abn ngsild:hasValue ?aantal_bike .
    ?s example:freeSlotNumber ?obj_slot .
    ?obj_slot ngsild:hasValue ?aantal_slot .
}
\end{minted}
\caption{SPARQL query met CONSTRUCT-clausule}
\label{code:construct}
\end{code}

In voorbeelden 1 en 2 werd bewust de `location'-eigenschap buiten beschouwing gelaten. Doordat de NGSI specificatie coördinaten in hun model implementeert aan de hand van GeoJSON is het niet mogelijk om met een SPARQL query individuele waarden te mappen op een specifieke URI. Dit komt doordat GeoJSON coördinaten bijhoudt in een array, terwijl het niet mogelijk is met een SPARQL query elementen uit een array te selecteren op basis van hun index. JSON-LD 1.1\footnote{\url{https://www.w3.org/TR/json-ld11/}} lost dit deels op met GeoJSON-LD\footnote{\url{https://geojson.org/geojson-ld/}}.
Een andere oplossing hiervoor is gebruikmaken van de RML.io toolchain uit sectie \ref{sec:rml} dat wel overweg kan met arrays en hun indexen.

\subsection{Voorbeeld 3}
\label{subsec:vb3}
In voorbeelden 1 en 2 werd telkens een gegevensset volgens het NGSI model gebruikt. In dit voorbeeld is het de beurt aan de GBFS specificatie met een gegevensset uit bijlage A: donkey.json. Om op deze set een SPARQL query te kunnen uitvoeren moet er opnieuw een @context worden toegevoegd zodat de JSON getransformeerd wordt naar JSON-LD. 
Deze keer voegen we aan de @context het `@vocab'-sleutelwoord toe. Dit sleutelwoord zorgt er voor dat er een default vocabularium aan de context wordt toegevoegd. Alle eigenschappen en types, waarvoor geen andere URI in de context werd gespecificeerd, zullen dezelfde prefix krijgen.
De sleutels in de JSON worden allen gemapt op een default URI met de prefix die werd meegegeven aan het @vocab-sleutelwoord, aangevuld met hun oorspronkelijke sleutel. Zonder meer kan deze JSON-LD nu worden gequeryd met volgende SPARQL query.

\begin{code}
\centering
\begin{minted}[breaklines]{sparql}
PREFIX default: <http://example.org/rdf#>

SELECT ?s ?id ?items ?docks WHERE {
    ?s default:fields ?fields .
    ?fields default:station_id ?id .
    ?fields default:num_bikes_available ?items .
    ?fields default:num_docks_available ?docks .
}
\end{minted}
\caption{SPARQL met NGSI-LD default context}
\label{ngsild-default-context}
\end{code}

Opnieuw kan er met een aanvullende CONSTRUCT-clausule een gegevensset geconstrueerd worden volgens het gewenste RDF-model.

\section{De mogelijkheden opgelijst}
Met de RML.io toolchain kunnen vanuit verschillende gegevensformaten (CSV, JSON, XML) RML regels gegenereerd worden. In een tweede stap wordt met behulp van die RML regels Linked Data gegenereerd. In het YARRRML document kunnen functies worden toegevoegd aan subject, predicaten en objecten die de gegevens kunnen manipuleren voordat het naar een LD gegevensset gemapt wordt. In bijlage C - velo.yaml wordt er zo een functie gebruikt.

Een simpelere manier, maar met beperktere mogelijkheid wegens het gebrek aan functies, is JSON-LD. Iedere JSON gegevensbron kan relatief gemakkelijk worden getransformeerd naar JSON-LD waardoor het te queryen is met SPARQL queries. In de @context kunnen de sleutels uit de JSON-objecten gemapt worden op URIs. De context kan op verschillende manieren met verschillende doeleinden worden opgebouwd door gebruik te maken van:

\begin{itemize}
    \item \textbf{default URIs} met behulp van het @vocab-sleutelwoord die daarna naar een definitief RDF-model geconstrueerd worden met een CONSTRUCT-query
    \item \textbf{RDF-model URIs} zodat het direct te publiceren is als LD gegevensset
    \item \textbf{NGSI-LD core context} wanneer een NGSI gegevensmodel wordt gebruikt. Deze context mapt de meta-gegevens van de object-eigenschappen naar gepaste URIs die daarna door middel van een CONSTRUCT-query naar het gewenste RDF-model kunnen worden gemapt
\end{itemize}

Afhankelijk van het formaat van de aangeleverde gegevensset en de eventuele manipulaties die er op moeten gebeuren is het tranformeren naar JSON-LD een snellere manier om de stap naar Linked Data te maken dan het gebruik van RML.
\chapter{Conclusie}
\label{chap:conclusie}
Het OSLO-programma van de Vlaamse overheid biedt RDF-vocabularia waarin gegevens worden gemodelleerd en gepubliceerd als Linked Open Data. Dit zorgt voor interoperabiliteit tussen de gegevens van honderden publieke sector diensten. Geld en tijd worden bespaard doordat de diensten gegevens kunnen uitwisselen tussen elkaar gebruikmakend van hun eigen business proces en ICT systemen.

OSLO mobiliteit: trips \& aanbod is een OSLO-vocabularium dat zich focust op personen die reizen en de mobiliteitsdiensten die ze daarvoor ter beschikking hebben. Daarmee is het de ideale kandidaat om connecties mee te maken met een nieuw vocabularium.
In hoofdstuk~\ref{chap:ontologie_voor_fietsdeeloperatoren} werd voor dit OSLO-vocabularium een uitbreiding, in plaats van een compleet nieuw vocabularium, gemaakt zodat ook fietsdeeloperatoren hun gegevens in dit model kunnen publiceren. 
Dankzij de abstracte implementatie, geleend uit de GBFS-specificatie, kunnen ook andere types vervoersmiddelen in de deeleconomie hun gegevens modelleren met dit vocabularium.

Om de operatoren zo weinig mogelijk te belasten met het (her)publiceren van de gegevens, worden er manieren aangereikt om die gegevens te gaan mappen (hoofstuk \ref{chap:on-boarding}).
De meeste gegevenssets van fietsdeeloperatoren worden gepubliceerd in een JSON-formaat gemodelleerd volgens de GBFS of NGSI specificatie. Door de gegevens te transformeren naar JSON-LD is ze sneller te herpubliceren als LD dan met de RML-methode. Om via de RML-toolchain naar LD te mappen, moet er eerst een YARRRML document geschreven worden die RML-regels genereert. De snelste manier om te transformeren naar JSON-LD is door een default vocabularium toe te voegen aan de context met het @vocab-sleutelwoord. Er moeten dan geen extra URI's per term worden gedefinieerd. Met een SPARQL query met CONSTRUCT-clausule kan dan een Linked Data gegevensset worden geconstrueerd met het gewenste RDF-model. Voor iedere specificatie, zoals GBFS of NGSI, kan een op voorhand gedefinieerde SPARQL CONSTRUCT-query worden geschreven zodat van de JSON-LD gegevensset een OSLO-conforme gegevensset kan worden gemaakt. De datapublishers hoeven dan enkel nog een JSON-context expliciet of via een HTTP Link header toe te voegen aan hun gegevensset.
Wanneer het oorspronkelijke gegevensformaat geen JSON is, zal er toch moeten worden teruggegrepen naar de RML-toolchain.

Deze stappen maken deel uit van het on-boardingsproces van gegevens naar de pijplijn van Vlaamse, OSLO-conforme, gegevens. Eens de gegevens in deze pijplijn zitten, kunnen ze geconsumeerd, gemanipuleerd voor andere doeleinden en gearchiveerd worden. Zo moet er geen geld meer besteed worden aan diensten die de gegevens laten werken, maar zullen de gegevens werken voor efficiëntere diensten.
\include{chapters/Bibliografie}
\begin{appendices}
\section*{Bijlage A - donkey.json}
\label{app:donkey.json}
Snipit van een Donkey Republic gegevensset in JSON-formaat gehaald uit de open database van stad Gent\footnote{\url{https://data.stad.gent/explore/dataset/donkey-republic-beschikbaarheid-deelfietsen-per-station}}. Deze array bevat twee objecten die elk informatie over de status van dat station bijhouden.
\begin{code}
\begin{minted}[breaklines]{json}
[
    {
        "datasetid": "donkey-republic-beschikbaarheid-deelfietsen-per-station",
        "recordid": "62d9aa911167ba3f491f93f7ce7adfb530925c8f",
        "fields": {
            "last_reported": "1600650631",
            "num_docks_available": 2,
            "is_renting": 1,
            "station_id": "12419",
            "is_installed": 1,
            "num_bikes_available": 1,
            "is_returning": 1
        },
        "record_timestamp": "2020-12-21T19:30:03.126+01:00"
    },
    {
        "datasetid": "donkey-republic-beschikbaarheid-deelfietsen-per-station",
        "recordid": "fbbd057f9b0f507a7bd9b6200c3a71e2a12bbc97",
        "fields": {
            "last_reported": "1607735889",
            "num_docks_available": 0,
            "is_renting": 1,
            "station_id": "16619",
            "is_installed": 1,
            "num_bikes_available": 3,
            "is_returning": 1
        },
        "record_timestamp": "2020-12-21T19:30:03.126+01:00"
    }
]
\end{minted}
\end{code}

\section*{Bijlage B - velo.json}
\label{app:velo.json}
Snipit van een gegevensset met Antwerpse Velo's, in JSON-formaat gehaald uit de 
\begin{code}
\begin{minted}[breaklines]{json}
[
    {
        "id": "BikeHireDockingStation:Antwerpen:035",
        "type": "BikeHireDockingStation",
        "address": {
            "type": "PostalAddress",
            "value": {
                "streetAddress": "cockerillkaai",
                "postalCode": "2000",
                "addressCountry": "BE"
            },
            "metadata": {}
        },
        "areaServed": {
            "type": "Text",
            "value": "Antwerpen",
            "metadata": {}
        },
        "availableBikeNumber": {
            "type": "Number",
            "value": 8,
            "metadata": {
                "timestamp": {
                    "type": "DateTime",
                    "value": "2020-12-22T08:32:12.00Z"
                }
            }
        },
        "freeSlotNumber": {
            "type": "Number",
            "value": 28,
            "metadata": {
                "timestamp": {
                    "type": "DateTime",
                    "value": "2020-12-22T08:32:12.00Z"
                }
            }
        },
        "location": {
            "type": "geo:json",
            "value": {
                "type": "Point",
                "coordinates": [
                    4.3877,
                    51.210318
                ]
            },
            "metadata": {}
        }
]
\end{minted}
\end{code}

\section*{Bijlage C - velo.yaml}
\label{app:velo.yaml}

\begin{code}
\begin{minted}[breaklines]{yaml}
prefixes:
  ex: 'https://example.be#'
  adr:  'https://data.vlaanderen.be/ns/adres#'
  dienst: 'https://stijnbrysbaert.github.io/OSLO-extension/mobiliteitsdiensten.ttl#'
  dc:   'http://purl.org/dc/terms/'
  ext:  'https://stijnbrysbaert.github.io/OSLO-extension/vocabulary.ttl#'
  foaf: 'http://xmlns.com/foaf/0.1/'
  geo:  'http://www.w3.org/2003/01/geo/wgs84_pos#'
  geonames: 'http://www.geonames.org/ontology#'
  grel: 'http://users.ugent.be/~bjdmeest/function/grel.ttl#'
  idlab-fn: 'http://example.com/idlab/function/'
  locn: 'http://www.w3.org/ns/locn#'
  mv:   'http://schema.mobivoc.org/'
  prov: 'http://www.w3.org/ns/prov#'
  rdf:  'http://www.w3.org/1999/02/22-rdf-syntax-ns#'
  rdfs: 'http://www.w3.org/2000/01/rdf-schema#'
  tn:   'https://data.vlaanderen.be/ns/transportnetwerk#'
  trips: 'https://data.vlaanderen.be/ns/mobiliteit/trips-en-aanbod#'
  xsd:  'http://www.w3.org/2001/XMLSchema#'

mappings:
  station:
    sources:
      - ['velo.json~jsonpath', '$.velos[*]']
    s: ex:$(id)
    po:
      - [a, [ext:Station, dc:Location]]
      - [rdfs:label, $(name.value)]
      - p: locn:geometry
        o:
          mapping: location
          condition:
            function: equal
            parameters:
              - [str1, $(id)]
              - [str2, $(id)]
      - p: 'trips:Transportobject.beschikbaarheid'
        o: 
          mapping: status
          condition:
            function: equal
            parameters:
              - [str1, $(id)]
              - [str2, $(id)]

  location:
    sources:
      - ['velo.json~jsonpath', '$.velos[*]']
    s: ex:location_$(id)
    po:
      - [a, geo:Point]
      - [geo:long, '$(location.value.coordinates[0])']
      - [geo:lat, '$(location.value.coordinates[1])']

  voertuigenBeschikbaar:
    sources:
      - ['velo.json~jsonpath', '$.velos[*]']
    s: ex:itemsAvailable_$(id)
    po:
      - [a, ext:VoertuigenBeschikbaar]
      - [ext:aantal, $(availableBikeNumber.value)]
      
  docksBeschikbaar:
    sources:
      - ['velo.json~jsonpath', '$.velos[*]']
    s: ex:docksAvailable_$(id)
    po:
      - [a, ext:DocksBeschikbaar]
      - [ext:aantal, $(freeSlotNumber.value)]

  status:
    sources:
      - ['velo.json~jsonpath', '$.velos[*]']
    s: ex:status_$(id)
    po:
      - [a, trips:Beschikbaarheid]
      - p: ext:voertuigTypesBeschikbaar
        o:
          mapping: voertuigenBeschikbaar
          condition:
            function: equal
            parameters:
              - [str1, $(id)]
              - [str2, $(id)]
      - p: ext:voertuigDocksBeschikbaar
        o:
          mapping: docksBeschikbaar
          condition:
            function: equal
            parameters:
              - [str1, $(id)]
              - [str2, $(id)]
      - [prov:generatedAtTime, $(availableBikeNumber.metadata.timestamp.value)]
      - p: ext:actief
        o:
          value: "true"
          datatype: "xsd:boolean"
        condition:
          function: idlab-fn:equal
          parameters:
            - [grel:valueParameter, $(status.value)]
            - [grel:valueParameter2, "working"]
      - p: ext:actief
        o:
          value: "false"
          datatype: "xsd:boolean"
        condition:
          function: idlab-fn:notEqual
          parameters:
            - [grel:valueParameter, $(status.value)]
            - [grel:valueParameter2, "working"]
\end{minted}
\end{code}

\section*{Bijlage D - NGSI normalized}
Genormaliseerd voorbeeld van een NGSI resultaat\footnote{\url{https://fiware-datamodels.readthedocs.io/en/latest/Transportation/Bike/BikeHireDockingStation/doc/spec/index.html}}.
\begin{code}
\begin{minted}[breaklines]{json}
{
    "id": "Bcn-BikeHireDockingStation-1",
    "type": "BikeHireDockingStation",
    "status": {
        "value": "working"
    },
    "availableBikeNumber": {
        "value": 20,
        "metadata": {
            "timestamp": {
                "type": "DateTime",
                "value": "2018-09-25T12:00:00"
            }
        }
    },
    "freeSlotNumber": {
        "value": 10
    },
    "location": {
        "type": "geo:json",
        "value": {
            "type": "Point",
            "coordinates": [2.180042, 41.397952]
        }
    },
    "address": {
        "type": "PostalAddress",
        "value": {
            "addressCountry": "ES",
            "addressLocality": "Barcelona",
            "streetAddress": "Gran Via Corts Catalanes,760"
        }
    }
}

\end{minted}
\end{code}

\section*{Bijlage E - NGSI key-value paren}
Een versimpelde weergave van een NGSI resultaat\footnote{\url{https://fiware-datamodels.readthedocs.io/en/latest/Transportation/Bike/BikeHireDockingStation/doc/spec/index.html}}, meestal gebruikt voor consumers en daarom zeer handig in gebruik voor de use cases in deze scriptie.
\begin{code}
\begin{minted}[breaklines]{json}
{
    "id": "malaga-bici-7",
    "type": "BikeHireDockingStation",
    "name": "07-Diputacion",
    "location": {
        "coordinates": [-4.43573, 36.699694],
        "type": "Point"
    },
    "availableBikeNumber": 18,
    "freeSlotNumber": 10,
    "address": {
        "streetAddress": "Paseo Antonio Banderas (Diputación)",
        "addressLocality": "Malaga",
        "addressCountry": "España"
    },
    "description": "Punto de alquiler de bicicletas próximo a Diputación",
    "dateModified": "2017-05-09T09:25:55.00Z"
}
\end{minted}
\end{code}

\end{appendices}

\end{document}
