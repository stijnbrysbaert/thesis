Abstract

% Context
There are different ways to publish mobility related data causing these datasets to be non-interoperable.
% Need
A unified data model that fosters creation of reusable identifiers for mobility infrastructure components, is required to increase data interoperability.
% Task
Following the principles of Linked Data, a proposal for an extensible RDF vocabulary that captures the straightforward object properties of bicycle sharing data is designed.
% Object
The vocabulary is focusing on the bicycle sharing use case so data of Flemish bicycle sharing operators can be republished as Linked Data.
% Findings
The OSLO mobiliteit: trips \& aanbod ontology of the Flemish government is a candidate to expand with properties defined in the GBFS specification to satisfy the needs of operators like Blue Bike and Velo.
% Conclusion
Mobility operators can republish their data in our developed vocabulary, but some of them can't invest a lot of time and money in republishing their data in a provided data model. Therefore, an intermediate party is needed to republish their data as Linked Data.
% Perspective
During following phases of research, the RDF data model is ready to be expanded to future mobility operators and become a single, all-encompassing and interoperable data model for mobility related data.