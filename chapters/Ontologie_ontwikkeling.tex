\chapter{Een RDF ontologie voor gegevens van  fietsdeeloperatoren}
\label{chap:ontologie_voor_fietsdeeloperatoren}
\Glspl{mobop} publiceren hun gegevens in verschillende gegevensmodellen en -formaten waardoor die gegevenssets niet interoperabel zijn met andere gegevens van de Vlaamse overheid. Het OSLO mobiliteit: trips \& aanbod vocabularium zou een uitbreiding moeten krijgen zodat we gegevens van Vlaamse fietsdeeloperatoren in het OSLO programma kunnen krijgen. GBFS biedt de nodige object eigenschappen en structuur die nodig zijn voor een model voor gegevens rond deelmobiliteit. 

De secties die volgen beschrijven hoe de uitbreiding op OSLO mobiliteit: trips \& aanbod als proof of concept tot stand zijn gekomen. De uitbreiding is niet volledig, maar generiek zodat er aan kan worden verder gebouwd indien het vocabularium wordt opgenomen in OSLO.
In tabel \ref{tab:oslo_prefixes} staan enkele OSLO-prefixes waarnaar zal worden verwezen in onderstaande secties. De technische informatie van iedere RDF-klasse waarnaar wordt gerefereerd kan worden teruggevonden door de URI te volgen gevormd door de prefix geconcateneerd met de RDF-term.

\begin{table}[]
\centering
\caption{gebruikte prefixes}
\label{tab:oslo_prefixes}
\begin{tabular}{ll}
ext:     & \url{https://stijnbrysbaert.github.io/OSLO-extension/vocabulary.ttl#} \\
dct   & \url{http://purl.org/dc/terms/} \\
mob   & \url{https://data.vlaanderen.be/ns/mobiliteit-trips-en-aanbod#} \\
net   & \url{https://data.vlaanderen.be/ns/netwerk#}         \\
tn    & \url{https://data.vlaanderen.be/ns/transportnetwerk#} \\
rdf   & \url{https://www.w3.org/1999/02/22-rdf-syntax-ns#} \\
weg   & \url{https://data.vlaanderen.be/ns/weg#}
\end{tabular}
\end{table}

\section{Station}
Allereerst is er nood aan een `Station'-klasse dat instanties van stations omschrijft. Een station in deze context is een object in het mobiliteitslandschap waar vervoermiddelen kunnen geparkeerd en gebruikt worden. Afgeleid uit het GBFS station status gegevensmodel\footnote{\url{https://github.com/NABSA/gbfs/blob/master/gbfs.md\#station_statusjson}} kan een station de real-time beschikbaarheid van voertuigen weergeven. In de GBFS specificatie wordt de capaciteit en beschikbaarheid van voertuigen gemodelleerd zodat het mogelijk is die eigenschappen bij te houden voor meerdere verschillende soorten vervoermiddelen: (elektrische)fiets, auto, step, ... Op die manier kan de gegevensset van een station worden aangevuld wanneer er een nieuw type voertuig wordt in ondergebracht.

tn:Transportpunt kan als superklasse worden gebruikt voor deze nieuwe klasse om een connectie te voorzien naar OSLO. Deze superklasse is zelf een subklasse van tn:Transportobject en tn:Netwerkelement en kan worden gelinkt aan een geografisch punt met het tn:geometrie predicaat.

\begin{table}[h]
\begin{tabular}{|l|l|}
\hline
\textbf{type}     & \textbf{klasse}                                                                                \\ \hline
URI               & {\color[HTML]{000000} ext:Station} \\ \hline
Specialisatie van & \begin{tabular}[c]{@{}l@{}}\url{https://data.vlaanderen.be/ns/transportnetwerk#Transportpunt}\\ \end{tabular} \\ \hline
Definitie         & Station waar voertuigen kunnen worden geplaatst en gebruikt.                                   \\ \hline
\end{tabular}
\end{table}

\section{VoertuigenBeschikbaar}
Het idee uit de GBFS specificatie om gegevens van verschillende vervoersmiddelen in eenzelfde station bij te houden, nemen we over naar deze OSLO uitbreiding. De klasse waarin het aantal beschikbare voertuigen, bijvoorbeeld deelfietsen, wordt gemodelleerd, moet generiek zijn. De VoertuigenBeschikbaar-klasse zal voor een specifiek vervoermiddel bijhouden hoeveel items er op dat moment beschikbaar zijn om te gebruiken. In dit domein kunnen twee eigenschappen worden aanvaard: aantal en dct:type. Van deze klasse kunnen er meerdere instanties gelinkt worden aan de station instantie.

\begin{table}[h]
\begin{tabular}{|l|l|}
\hline
\textbf{type}     & \textbf{klasse}                                                                                \\ \hline
URI               & {\color[HTML]{000000} ext:VoertuigenBeschikbaar} \\ \hline
Definitie         & Het aantal voertuigen van een bepaalt type dat beschikbaar is.                                   \\ \hline
\end{tabular}
\end{table}

\subsection{aantal}
Deze eigenschap duidt het aantal voertuigen van een bepaald vervoermiddel aan.

\begin{table}[h]
\begin{tabular}{|l|l|}
\hline
\textbf{type}     & \textbf{eigenschap}                                                                                \\ \hline
URI               & {\color[HTML]{000000} ext:aantal} \\ \hline
Definitie         & Aantal items van een instantie  \\ \hline
Domein & \begin{tabular}[c]{@{}l@{}}ext:VoertuigenBeschikbaar\\ext:VoertuigenDocks\end{tabular} \\ \hline
Bereik & \url{http://www.w3.org/2001/XMLSchema#integer} \\ \hline
\end{tabular}
\end{table}

\subsection{type}
De dct:type eigenschap verwacht volgens het applicatieprofiel van OSLO mobiliteit: trips \& aanbod een instantie van het rdf type Resourcetype. Resourcetype werd echter nog niet gedefinieerd in het applicatieprofiel. Het omschrijven van deze klasse valt buiten de scope van deze scriptie. 

% In het wegen-vocabularium\footnote{\url{https://data.vlaanderen.be/ns/weg}} bestaat er een eigenschap weg:voertuigtype dat ook ideaal lijkt om het type voertuig aan te duiden. Deze eigenschap kan helaas enkel worden toegepast worden in het weg:Rijrichting domein. Er zal een weloverwogen beslissing moeten worden gemaakt over welke RDF term zal worden 

\section{DocksBeschikbaar}
De klasse `DocksBeschikbaar' wordt gelinkt met dezelfde eigenschappen als de VoertuigenBeschikbaar-klasse: aantal en type. De klasse houdt het aantal \glspl{dock} dat beschikbaar is in een station voor een bepaalt vervoermiddel bij.

\section{Beschikbaarheid}
De beschikbaarheid-klasse gedefinieerd in OSLO mobiliteit: trips \& aanbod kan gelinkt worden aan een station en bevat instanties die betrekking hebben tot de mate waarin iets voorhanden is voor gebruik. Een station heeft exact één instantie van de beschikbaarheid-klasse en wordt met behulp van de mob:Transportobject.beschikbaarheid eigenschap gelinkt. Gezien een station een subklasse is van mob:Transportpunt en die klasse op zijn beurt weer een subklasse is van mob:Transportobject, wordt de mob:Transportobject.beschikbaarheid eigenschap binnen het juiste domein gebruikt.

Wat volgt zijn twee eigenschappen toepasbaar op de mob:Beschikbaarheid-klasse waarmee gegevens rond de beschikbaarheid van voertuigen en docks in een station kunnen worden gelinkt.

\subsection{voertuigTypesBeschikbaar}
Deze eigenschap vormt de relatie tussen de beschikbaarheids-instantie van een station met het real-time aantal beschikbare voertuigen van een bepaalt type.

\begin{table}[h]
\begin{tabular}{|l|l|}
\hline
\textbf{type}     & \textbf{eigenschap} \\ \hline
URI               & ext:voertuigTypesBeschikbaar \\ \hline
Definitie         & Het aantal voertuigen van een bepaalt type dat beschikbaar is.                                   \\ \hline
Domein & mob:Beschikbaarheid \\ \hline
Bereik & ext:VoertuigenBeschikbaar \\ \hline
\end{tabular}
\end{table}

\subsection{voertuigDocksBeschikbaar}
Deze eigenschap vormt de relatie tussen de beschikbaarheids-instantie van een station met het real-time aantal beschikbare docks voor een bepaalt type voertuig. 

\begin{table}[h]
\begin{tabular}{|l|l|}
\hline
\textbf{type}     & \textbf{eigenschap} \\ \hline
URI               & ext:voertuigDocksBeschikbaar \\ \hline
Definitie         & Beschikbaarheid van het aantal plaatsen specifiek voor een bepaalt type voertuig.  \\ \hline
Domein & mob:Beschikbaarheid \\ \hline
Bereik & ext:DocksBeschikbaar \\ \hline
\end{tabular}
\end{table}

\section{Voorbeeld}
Figuur~\ref{fig:rdf_gegevensset_voorbeeld} beeldt de graaf af die een mogelijke gegevensset, gepubliceerd volgens de OSLO uitbreiding, visualiseert. Dankzij de klasses VoertuigTypesBeschikbaar en DocksBeschikbaar kunnen er onbeperkt aantal instanties gelinkt worden die per type voertuig informatie bevat.

\begin{figure}[h]
	\centering
	\begin{subfigure}{\textwidth}
		\centering
		\centerline{
			\includegraphics[width=\textwidth]{images/rdf_dataset.pdf}
		}
	\end{subfigure}
	\caption{Gegevensset gebruikmakend van RDF model}
	\label{fig:rdf_gegevensset_voorbeeld}
\end{figure}