\newglossaryentry{maasaanbieder}
{
    name=MaaS-aanbieder,
    description={Een organisatie die een \acrshort{maas} platform aanbiedt}
}

\newglossaryentry{mobop}{
    name=mobiliteitsoperator,
    plural=mobiliteitsoperatoren,
    description={Een organisatie die van mobiliteit een service maakt en deze aanbiedt aan reizigers}
}

\newglossaryentry{deelmob}{
    name = deelmobiliteit,
    description = {Het delen van een fiets, auto, brommer, ... met andere mensen zodat kosten zoals aankoop en onderhoud kunnen verdeeld worden onder de gebruikers. Kan aangeboden worden als abonnement of pay as you go waarbij voor beide opties de kosten meestal oplopen naar gelang de tijd dat je het voertuig gebruikt}
}

\newglossaryentry{fietsdeelop}{
    name = fietsdeeloperator,
    plural = fietsdeeloperatoren,
    description = {Een combinatie van \gls{mobop} en \gls{deelmob} waarbij het vervoermiddel een fiets is}
}

\newglossaryentry{ontologie}{
    name = vocabularium,
    plural = vocabularia,
    description = {Een vocabularium definieert concepten en relaties (ook termen genoemd) om zaken binnen een bepaalt gebied te omschrijven en weer te geven~\footnote{\url{https://www.w3.org/standards/semanticweb/ontology}}}
}

\newglossaryentry{RDF vocabularium}{
    name = RDF vocabularium,
    description = {Zie \gls{ontologie.}}
}

\newglossaryentry{deeleconomie}{
    name = deeleconomie,
    description = {De deeleconomie is een socio-economisch systeem waarin delen en collectief consumeren centraal staat. Het gaat om gezamenlijk creatie, productie, distributie, handel en consumptie van goederen en diensten\footnote{\url{https://nl.wikipedia.org/wiki/Deeleconomie}}}
}

\newglossaryentry{dock}{
    name = dock,
    plural = docks,
    description = {Een gereserveerde plaats waar exact één voertuig in kan worden geparkeerd. In sommige gevallen is een dock enkel en alleen bruikbaar voor het parkeren van een specifiek vervoermiddel. Voertuigen van een ander type dan waarvoor het dock is bedoeld, kunnnen hier niet worden geparkeerd}
}