\thispagestyle{empty}    % Don't show page number

\begin{center}
\textbf{Dankwoord}
\end{center}

Deze masterproef is het eindwerk van mijn masteropleiding en daarmee ook het einde van mijn hoger onderwijs traject. Na het behalen van een professionele bachelor startte ik een schakeljaar industriële wetenschappen afstudeerrichting informatica om daar aansluitend de bijhorende masteropleiding af te leggen. Door dit te combineren met de rol als groepsleider van een scoutsgroep met 360 leden werd dit een intensieve periode van drie en een half jaar. Gedurende die periode stond ik er gelukkige nooit alleen voor. Ik kon rekenen op veel mensen rondom mij die me ondersteunden op vlak van zowel scouting als studies. De ruimte dat dit dankwoord mag innemen is helaas te klein om al die namen te vernoemen. Toch had ik graag mijn ouders vermeld en hun bedankt mij de mogelijkheid te geven te kunnen studeren wat, wanneer en hoe ik zelf wilde. Ik ben me ervan bewust dat dit een uitzonderlijke kans is. Dankjewel iedereen.

Deze masterproef kwam tot stand met de onmisbare hulp van een aantal personen. Allereerst had ik mijn promotoren prof. dr. ir. Ruben Verborgh en dr. Pieter Colpaert willen bedanken mij de kans te geven me te kunnen verdiepen in de interessante en veel belovende wereld van Linked Data. Daarbovenop op wil ik Pieter extra bedanken voor zijn enthousiasme en gedrevenheid in het onderwerp. Mijn doel iets interessant te maken van mijn masterproef werd ook zijn doel. Altijd beschikbaar en paraat met kritische ideeën en feedback. Ook mijn begeleiders Brecht Van de Vyvere, Dylan Van Assche, Julian Andres Rojas Melendez en Harm Delva wil ik bedanken voor hun vooral technische ondersteuning via ons mattermost kanaal op ieder moment van de dag. Speciale dank gaat uit naar mijn broer Maarten\footnote{\url{https://www.linkedin.com/in/maarten-brysbaert}} die me op zeer korte tijd een demo hielp in elkaar steken waarmee ik mijn contributies kon visualiseren tijdens presentaties.

Om af te sluiten had ik graag mijn vriendin Feuille Boi bedankt voor haar onvermoeibare steun, ook tijdens moment waarop dat niet zo vanzelfsprekend was. Ook mijn zussen die altijd blij waren me te zien als ik thuis kwam waren een steun. Daarnaast wil ik mijn huisgenoten Aileen Bonne en Wiebe Vermiesch bedanken voor de leuke momenten tussendoor en andere scoutsvrienden voor de ontspannende wandelingen mét social distancing uiteraard. Extra dank gaat uit naar Aileen voor de zelfreflectie momenten ten behoeve van mijn masterproef.

Bedankt allemaal!
\newline
Stijn Brysbaert