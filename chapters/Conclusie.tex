\chapter{Conclusie}
\label{chap:conclusie}
Het OSLO programma van de Vlaamse overheid biedt RDF vocabularia waarin gegevens worden gemodelleerd en gepubliceerd als Linked Open Data. Dit zorgt voor interoperabiliteit tussen de gegevens van honderden publieke sector diensten. Geld en tijd worden bespaard doordat de diensten gegevens kunnen uitwisselen tussen elkaar gebruikmakend van hun eigen business proces en ICT systemen.
OSLO mobiliteit: trips \& aanbod is één van die OSLO vocabularia. Dit vocabularium focust zich op personen die reizen en de mobiliteitsdiensten die ze daarvoor ter beschikking hebben. In hoofdstuk~\ref{chap:ontologie_voor_fietsdeeloperatoren} werd voor dit vocabularium een uitbreiding gemaakt zodat ook fietsdeeloperatoren hun gegevens in dit model kunnen publiceren. Dankzij de abstracte implementatie, geleend uit de GBFS specificatie, kunnen ook andere types vervoersmiddelen in de deeleconomie hun gegevens modelleren met dit vocabularium.

Aangezien operatoren moeilijk te overtuigen zijn hun gegevens direct te publiceren in een door de overheid verplicht gegevensmodel, moeten er tools worden voorzien die de gegevens mappen. Na het mappen dienen gegevens te worden herpubliceerd  als Linked Open Data. Dat mappen kan op een paar verschillende manieren beschreven in hoofstuk~\ref{chap:on-boarding}. De meeste gegevenssets van fietsdeeloperatoren worden gepubliceerd in een JSON-formaat gemodelleerd volgens de GBFS of NGSI specificatie. Door de gegevens te transformeren naar JSON-LD is ze sneller te herpubliceren als LOD dan met de RML-methode. Om via de RML-toolchain naar LD te mappen, moet er eerst een YARRRML document geschreven worden die RML-regels genereert. De snelste manier om te transformeren naar JSON-LD is door een default vocabularium toe te voegen aan de context met het @vocab-sleutelwoord. Er moeten dan geen extra URIs per term worden gedefinieerd. Met een SPARQL query met CONSTRUCT-clausule kan dan een Linked Data gegevensset geconstrueerd worden met het gewenste RDF-model. Wanneer het oorspronkelijke gegevensformaat geen JSON is, zal er toch moeten worden teruggegrepen naar de RML-toolchain.

Deze stappen maken deel uit van het on-boardingsproces van gegevens naar de pijplijn van Vlaamse, OSLO conforme, gegevens. Eens de gegevens in deze pijplijn zitten, kunnen ze geconsumeerd, gemanipuleerd voor andere doeleinden en gearchiveerd worden. Zo moet er geen geld meer besteed worden aan diensten die de gegevens laten werken, maar zullen de gegevens werken voor efficiëntere diensten.