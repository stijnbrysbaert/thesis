\chapter{Conclusie}
\label{chap:conclusie}
Het OSLO-programma van de Vlaamse overheid biedt RDF-vocabularia waarin gegevens worden gemodelleerd en gepubliceerd als Linked Open Data. Dit zorgt voor interoperabiliteit tussen de gegevens van honderden publieke sector diensten. Geld en tijd worden bespaard doordat de diensten gegevens kunnen uitwisselen tussen elkaar gebruikmakend van hun eigen business proces en ICT systemen.

OSLO mobiliteit: trips \& aanbod is een OSLO-vocabularium dat zich focust op personen die reizen en de mobiliteitsdiensten die ze daarvoor ter beschikking hebben. Daarmee is het de ideale kandidaat om connecties mee te maken met een nieuw vocabularium.
In hoofdstuk~\ref{chap:ontologie_voor_fietsdeeloperatoren} werd voor dit OSLO-vocabularium een uitbreiding, in plaats van een compleet nieuw vocabularium, gemaakt zodat ook fietsdeeloperatoren hun gegevens in dit model kunnen publiceren. 
Dankzij de abstracte implementatie, geleend uit de GBFS-specificatie, kunnen ook andere types vervoersmiddelen in de deeleconomie hun gegevens modelleren met dit vocabularium.

Om de operatoren zo weinig mogelijk te belasten met het (her)publiceren van de gegevens, worden er manieren aangereikt om die gegevens te gaan mappen (hoofstuk \ref{chap:on-boarding}).
De meeste gegevenssets van fietsdeeloperatoren worden gepubliceerd in een JSON-formaat gemodelleerd volgens de GBFS of NGSI specificatie. Door de gegevens te transformeren naar JSON-LD is ze sneller te herpubliceren als LD dan met de RML-methode. Om via de RML-toolchain naar LD te mappen, moet er eerst een YARRRML document geschreven worden die RML-regels genereert. De snelste manier om te transformeren naar JSON-LD is door een default vocabularium toe te voegen aan de context met het @vocab-sleutelwoord. Er moeten dan geen extra URI's per term worden gedefinieerd. Met een SPARQL query met CONSTRUCT-clausule kan dan een Linked Data gegevensset worden geconstrueerd met het gewenste RDF-model. Voor iedere specificatie, zoals GBFS of NGSI, kan een op voorhand gedefinieerde SPARQL CONSTRUCT-query worden geschreven zodat van de JSON-LD gegevensset een OSLO-conforme gegevensset kan worden gemaakt. De datapublishers hoeven dan enkel nog een JSON-context expliciet of via een HTTP Link header toe te voegen aan hun gegevensset.
Wanneer het oorspronkelijke gegevensformaat geen JSON is, zal er toch moeten worden teruggegrepen naar de RML-toolchain.

Deze stappen maken deel uit van het on-boardingsproces van gegevens naar de pijplijn van Vlaamse, OSLO-conforme, gegevens. Eens de gegevens in deze pijplijn zitten, kunnen ze geconsumeerd, gemanipuleerd voor andere doeleinden en gearchiveerd worden. Zo moet er geen geld meer besteed worden aan diensten die de gegevens laten werken, maar zullen de gegevens werken voor efficiëntere diensten.